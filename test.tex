\documentclass[2pt,landscape]{article}
%\usepackage{fontspec}
%\setsansfont{Roboto Condensed}
\usepackage{multicol}
\usepackage{calc}
\usepackage{ifthen}
\usepackage[landscape]{geometry}
\usepackage{amsmath,amsthm,amsfonts,amssymb}
\usepackage{color,graphicx,overpic}
\usepackage{hyperref}
\usepackage{gensymb}
\usepackage[sfdefault, condensed]{roboto}
\geometry{top=0.3cm,left=0.3cm,right=0.3cm,bottom=0.3cm}


% This sets page margins to .5 inch if using letter paper, and to 1cm
% if using A4 paper. (This probably isn't strictly necessary.)
% If using another size paper, use default 1cm margins.
%\ifthenelse{\lengthtest { \paperwidth = 11in}}
%    { \geometry{top=.5in,left=.5in,right=.5in,bottom=.5in} }
%    {\ifthenelse{ \lengthtest{ \paperwidth = 297mm}}
%        {\geometry{top=1cm,left=1cm,right=1cm,bottom=1cm} }
%        {\geometry{top=1cm,left=1cm,right=1cm,bottom=1cm} }
%    }

% Turn off header and footer
\pagestyle{empty}

% Redefine section commands to use less space
\makeatletter
\renewcommand{\section}{\@startsection{section}{1}{0mm}%
	{-1ex plus -.5ex minus -.2ex}%
	{0.5ex plus .2ex}%x
	{\normalfont\large\bfseries}}
\renewcommand{\subsection}{\@startsection{subsection}{2}{0mm}%
	{-1explus -.5ex minus -.2ex}%
	{0.5ex plus .2ex}%
	{\normalfont\normalsize\bfseries}}
\renewcommand{\subsubsection}{\@startsection{subsubsection}{3}{0mm}%
	{-1ex plus -.5ex minus -.2ex}%
	{1ex plus .2ex}%
	{\normalfont\small\bfseries}}
\makeatother

% Define BibTeX command
\def\BibTeX{{\rm B\kern-.05em{\sc i\kern-.025em b}\kern-.08em
		T\kern-.1667em\lower.7ex\hbox{E}\kern-.125emX}}

% Don't print section numbers
\setcounter{secnumdepth}{0}


\setlength{\parindent}{0pt}
\setlength{\parskip}{0pt plus 0.5ex}

%My Environments
\newtheorem{example}[section]{Example}
% -----------------------------------------------------------------------

\begin{document}
	\raggedright
	\footnotesize
	\begin{multicols*}{5}
		
		
		% multicol parameters
		% These lengths are set only within the two main columns
		%\setlength{\columnseprule}{0.25pt}
		\setlength{\premulticols}{1pt}
		\setlength{\postmulticols}{1pt}
		\setlength{\multicolsep}{1pt}
		\setlength{\columnsep}{2pt}
		
		\begin{center}
			\Large{\underline{Sensor Principles}} \\
			
		\end{center}
		
		
		% Summary starts here
		% --------------------------------------------------------------------------------------







Integrated Interface Circuits \textbullet 
Errors and Noise


Jens Anders
Institute for Theory of Electrical Engineering 
Department of Electrical Engineering and Computer 
Science
University of Stuttgart









Integrated Interface Circuits
\textbullet 
Introduction

Jens Anders
Institute of Smart Sensors
Department of Engineering , Computer Science 
University of Stuttgart



?	Lectures:
\textbullet 	Mondays: 11:30 AM until 1 PM
?	Seminar at the end of the lecture period/beginning of the lecture free
period
?	Participation in the seminar is mandatory for taking the exam
?	The exam will be a written exam



?	Introduction/Motivation
\textbullet 	Why/where do we need sensors?
\textbullet 	What are the challenges/problems in reading out the sensors?
?	Noise and other sources of error
?	Categorizing different types of sensor
\textbullet 	Self-generating vs. modulating sensors
\textbullet 	Electrical behavior
?	Readout Concepts
\textbullet 	Absolute vs. differential measurements
\textbullet 	AM vs. FM detection
\textbullet 	Resonant readouts
\textbullet 	Phase sensitive-detection (Lock-in amplifiers)
\textbullet 	Continuous-time vs. discrete time readout



?	Readout Concepts (cont.)
\textbullet 	Open-loop vs. Closed loop measurements
?	Readout Circuit Implementations
\textbullet 	Instrumentation amplifiers
\textbullet 	Transimpedance amplifiers
\textbullet 	Switched-capacitor readout circuits
?	Data-converters for sensor readouts
\textbullet 	A/D converters
\textbullet 	D/A converters
\textbullet 	T/D converters (if time permits)



?	Medical applications (1/2): MRI, Ultrasound
www.medical.siemens.com	                    
http://www.magnet.fsu.edu/	www.sciencephoto.com
http://www.healthcare.philips.com	http://www.nibib.nih.gov/	http://www.gehealthcare.com/



?	Medical applications (2/2): CT, PET/CT


http://www.toshiba-medical.eu/

http://www.healthcare.philips.com


http://www.cyberphysics.co.uk/	http://neuros.net/

http://www.uib.no	http://www.siemens.com/



?	Automotive applications




http://www.toyota.co.uk/


http://auto.howstuffworks.com


http://newsimages.motoring.co.uk/






http://www.eetimes.com/
http://i.telegraph.co.uk/



?	Consumer electronics

http://www.arj.no/2007/01/14/iphone-sensing/	http://www.kinectingforwindows.com/

http://pertemi.wordpress.com/	http://www.touchuserinterface.com/



?	Applications in Science/Metrology

















http://www.bruker.com


http://simple.wikipedia.org


Source: Prof. Keller, Uni 
Graz











Array of Photodiodes	Pixel A/D converter





?	Sensors sense their environment
?	Sensors can produce electrical or non-electrical signals 
transduction might be required
?	Interface electronics typically, amplify, frequency convert, filter and A/D
convert the sensed signal

























REVIEW
OPAMP BASICS







?	Maximum input voltage in open loop configuration:



A ? 105, VDD ? ?VSS ? 10 V,V


? 18 V ==> V	? V


/ A ? 90 ?V  


0	out,range	in,max	out,range	0
?	Opamps should be used in closed loop configuration
?	Tradeoff between open loop gain and accuracy





?	When using the opamp in feedback, stability becomes a concern!
?	Stability criterion: Simplified Nyquist criterion (negative feedback)
loop gain: ??	= ?? ? ????



phase margin: ??


= 1 ? PM = 180\textbullet 
+ ?






gain margin: ?


= ?180\textbullet ? 
GM = 20 ? 
log10





?	Assuming ???? ? ?


Vx ?


? ?Vout


? ? ? A ?? ? ? 
?Vin


?Vx ? ?


? ? A ?? ? ? ?Vin


?	? ?Vout ?



? 


?	Very basic feedback theory
\textbullet 	Ao(?) is the forward gain (open loop gain)
\textbullet 	? is the feedback gain
\textbullet 	?\textbullet Ao(?) is the loop gain
\textbullet 	1 + ?? ? ????: feedback factor
\textbullet 	Closed-loop TF:


Ao(?)
Vin	Vout


?	There is a tradeoff between open loop 
gain and DC accuracy.
?	Tradeoff of gain vs. bandwidth.
?	Tradeoff of input and output resistance. 
(depending on the type of feedback)
Buffer example of DC tradeoff:




Open loop response	Closed loop response (?? = 1)
?	Placing the first non-dominant pole before or not to close behind the 
GBW results in peaking in the amplitude response of the closed loop 
system





?	In the time-domain, placing the non dominant pole below or not 
sufficiently far beyond the GBW manifests itself as ringing in the step 
response
?	Note: Depending on the application, the frequency and/or time-domain
behavior is more important



Loop gain ?? ? ????	Closed loop gain ????/(1 + ?? ? ????)
?	The worst case in terms of stability occurs for ?? = 1 because 
(assuming a passive feedback network) this choice maximizes the loop 
gain




?	Assuming a finite value for ??out and a very large open loop gain ??0

?	How is this condition ensured in the circuit?

The opamp in the negative feedback loop adjusts ??out  appropriately
?	The potential at the opamp\textbullet s noninverting input is virtually copied to 
the inverting terminal (virtual short)





?	Due to the virtual short between the inverting and noninverting opamp 
inputs, we have:



?	The fact that typical opamps have large differential input impedances
further attenuate the opamp\textbullet s input current





?	Apply KCL to node X




























?	Feedback factor:


T ?? ?


? Vx? ?
Vx


R1	? A ?? ?


?	Closed loop TF:	? ?
\textbullet 	Either apply KCL at node X or use the finding of slide 16








?	If negative feedback is applied around an opamp, the following rules
apply:

?	Important: The rules only apply if the opamp operates linearly (e.g. no 
clipping or oscillation at its output)




?	Note: The opamp is used in positive feedback here (the output is fed
back to the noninverting input)
?	What do we know? ???? ? ? ? ?????? ? 0 for finite ???? ?


?	Assume that initially ???? is positive and not too large (i.e. ?? 
VDD) and that the output is at zero volts


? ???? <


??out = ??	? ???? ? ??out becomes positive as well
the constraint ??in = ??FB  enforces that ???? further increases
??out and ???? increase together until the opamp rails





?	Virtual short (???? = ??out) and KCL at nodes X and Y


0 ? Vin


?Vx


? ?V


?V ? ? sC


?	?Vout


?Vx ?


? 
R	out	x	1	R
1	2


0 ? Vout


?Vx ? V


? sC


out	2
2







?	Calculate the input impedance of the circuit above!



Zin


? vin
i


in
?	Hint: Assume an ideal opamp and a voltage drive





















?	Voltage at node 1: ????1	in
?	Ideal opamp enforces virtual short: ??X3



= ??X1



= ??in


?	Ideal opamp: Current through ??1 also flows through ??2:


??X2


= ??x3 ?
??1


= 
??in 
?
??
1





?	Ideal opamp: Current through ??conv


is ??in


: ??in


= (??in???x2)
??conv










?	The circuit should not be used with a current drive
?	Voltage drive	negative feedback
?	Current drive	positive feedback
?	Intuitive explanation: For current drive there is unity gain feedback to
the noninverting input and less than unity gain feedback to the inverting









Integrated Interface Circuits \textbullet 
Errors and Noise


Jens Anders
Institute for Theory of Electrical Engineering 
Department of Electrical Engineering and Computer 
Science
University of Stuttgart



?	Limit of detection (LOD) definition:
The minimum level of the input quantity which can reproducibly 
be measured at a specified SNR (typically: ????NR = ???? ????B)
?	Practical signals are specified by their dynamic range (DR):
DR is the ratio between the maximum and minimum signal levels 
occurring within inaccuracy levels



?	Two different types of error:
\textbullet 	Deterministic errors
e.g. source loading, offset, gain error
\textbullet 	Random (stochastic) errors
e.g. thermal noise, 1/f noise
?	Errors can be quantified as absolute errors and relative errors




Absolute error: ??? =


???


? ??0




Relative error:	=




where ??0 is the true value and


???


is the measured quantity/the 
estimate



?	The maximum inaccuracy ???max should be specified such that the
true value lies within the interval [??? ? ???max, ??? + ???max]






?	Assumption:
The output quantity ?? is affected by the ?? parameters ??????=1..?? according 
to
y ? f ?x1, x2,\textbullet , xN ?

?	Then, assuming small deterministic fluctuations in the ?????? the total error


in ?? is given by


?y ?




?
i ?1


?f ? ?x
?xi


?	Consequently, if we denote by ???????,max  the (small) maximum inaccuracy 
of each ?????? , then ??? is bounded by



?ymax




? ?
i ?1


?f 
?xi



? ?x




i ,max






?	The partial derivative ????????
??????????


is called sensitivity w.r.t. ??????





?	If the quantity to be measured depends on the 3 parameters ??in, ??error,1
and ??error,2 according to


y ? G ? xin


?	k1 ? xerror,1 ? k2 ? xerror,2



and the uncertainties in ??in, ??error,1 and ??error,2 are 0 (??in is the perfectly 
known input signal), ???error,1 and ???error,2, the uncertainty in y becomes



?y ?


? ?G ? xin


?	k1 ? 
xerror,1 ? 
k2 ? 
xerror,2 ? ? 
?x
?x



error,1 ?


? ?G ? xin


?	k1 ? 
xerror,1 ? 
k2 ? 
xerror,2 ? ? 
?x
?x





error,2


error,1	error,2
? k1 ? ?xerror,1 ? k2 ? ?xerror,2
?	Typical sources of additive errors are:
\textbullet 	Offset
\textbullet 	Additive noise (cave: here rms values need to be considered)
?	The uncertainty in the measurand due to additive errors is best 
specified as an absolute error



?	Assume that the output ??(??) depends on the input Xin and the error 
source ??error according to
Y ?? ? ? H ??, xerrror ? ? Xin ?? ?
?	Then, the relative error in ??(??) becomes:


?Y ?? ?


?  	1	 ? ? ??H ??, xerror ? ? Xin ?? 
??? ? ?x




Y ?? ?


Y ?? ?


?xerror


error


?	Xin ?? ?	? ?H ??, xerror ? ? ?x


H ??, x	? ? X ?? ?	?x


error


error	in	error
?	1	? ?H ??, xerror ? ? ?x


H ??, x	?	?x


error


error	error

which is independent of Xin
?	Therefore, the effect of multiplicative errors is best described as a 
relative error in the output signal



?	Interference:
\textbullet 	Interference is caused by an undesired coupling of external signals into the 
measurement setup
\textbullet 	Example: 50 Hz hum
?	Noise:
\textbullet 	Noise is caused by random fluctuations originating in the measurement 
setup itself
\textbullet 	Example: thermal noise
?	Both noise and interference can usually be modeled as additive error 
signals
they can be modeled by additional error sources in the system
?	Cave: noise and interference should not be confused with signal 
distortion, which is caused by intrinsic system nonlinearities in either 
the amplitude or the frequency domain


Signal @ 1 kHz




?	Total signal: ??


= 0.1 ? cos


+ 2 ? cos


+ ??(??)





?	??


has 
a 
varia
nce 
of ???? 
= 
0.01


?	The signal is barely visible in the time domain, but easy to see in the 
frequency domain



?	Errors are introduced at multiple points along the signal processing 
chain

?	Examples of such error sources are offset, distortion, noise, crosstalk, 
cross sensitivity, gain error \textbullet 
?	Question: What is the best way to assess their combined effect on the 
measured signal?



?	General idea: We can only compare the effect of two different error sources 
when they are referred to the same point along the signal processing chain
?	Two natural choices for the comparison points are the overall input and the
overall output (input allows for an intuitive comparison with the measurand)







?	For linear(ized) systems the superposition principle applies:
Refer all error sources to the output and then refer them to the input by dividing 
by the overall signal gain
?	Deterministic sources of error:


?	Random uncorrelated sources of error:



?	Output referred error:
The effect of an error source on the overall output is called the output 
referred error
?	Input referred error:
The equivalent effect of an error source on the overall input is called 
the (equivalent) input referred error






?	Example for finding the equivalent input and output referred noise: 
Error source: ????2
Find TF from ????2  to the output ??: ????,??2	? ??out,??2   = ????,??2  ? ????2
Refer the result back to the input using the overall TF ????,??:


????2


= ??out,??2
????,??


= ????,??2


?	????2
????,??



?	Fluctuations of a physical quantity can be mathematically modeled as a
random or stochastic process
?	A random process (r.p.) can be seen as a family of functions ??
where ??(?? = ??0) is a random variable with 1st-order probability density 
function (pdf) ????(??, ??)


?	The instantaneous amplitude of the signal ??
only be predicted with probability ????(??, ??0)


at time ??0 can therefore









?	The mean or expected value ????(??) is obtained by averaging the amplitude ??
at time ?? over all possible realizations of ??(??)
?


mx (t ) ? E ?x(t )?  


? x ? px (x,t )dx
??


?	The autocorrelation function (ACF) is a measure of the statistical


dependence between the random variables ??
epochs


and ??(??2) at two different


Rx (t1,t2 ) ? E[x(t1 )x(t2 )]  


?   ?
? ?
?? ??


x1 ? x2


? px (x1, x2,t1,t2 )dx1dx2





where ????(??1, ??2, ??1, ??2) is the 2nd-order pdf of ??(??)
?	The autocovariance is defined by

Cx (t1,t2 ) ? E ???x(t1 ) ? mx (t1 )??x(t2 ) ? mx (t2 )???  





Rx (t1,t2 ) ? mx (t1 ) ? mx (t2 )



?	A random process is called wide sense stationary (WSS) if its mean is 
independent of ?? and its ACF only depends on the time difference ?? = ??2 ? ??1



mx (t ) ? mx


? const.


Rx (t1,t2 ) ? Rx (t2 ? t1 ) ? Rx (? )


C (t ,t


) ? C (t ? t ) ? C


(? ) ? R (? ) ? m2


x	1	2


x	2	1


x	x	x


?	The normalized ACF (aka correlation coefficient) is defined as


? (? ) ? Cx (? ) ,	where ? 2 ? C



(? ? 0) is the so-called variance


x	2	x	x
x
?	For real random processes one can show that:


Rx (? ) ? Rx (?? ),	Cx (? ) ? Cx (?? )
C (0) ? ? 2,	R (0) ? ? 2 ? m2,	? (0)=1
x	x	x	x	x	x



?	The Power Spectral Density (PSD) of a WSS r.p. x(t) is the Fourier transform 
of its ACF


Sx (f ) ? F ?Rx


(? )? ?


?
? Rx
??
?


(? ) ? e? j 2?f? d?


R (? ) ? F ?1?S


(f )? ?


? Sx
??


(f ) ? e? j 2?f? df





?	PSD has units of [V2] if ??(??) is a voltage and [A2]



if ??(??) is a current


Hz	Hz
?	The above definition defines the so-called double-sided PSD
?	Most instruments display the so-called single-sided PSD which is related to the 
double-sided PSD according to



S? (f ) ?


?2Sx (f )	for f ? 0
?


?Sx (f )	for f ? 0



?	The total power associated with the r.p. is given by
?


Px ?


? Sx (f ) ? df
??


? Rx (0)



?	Note: By definition the rms value is the square root of ????
?	For a noise voltage PSD
?	?


2
n,rms


?	S (f ) ? df ?
n
??	0


?	(f ) ? df
n



?	For a noise current PSD


?	?


2
n,rms


?	S (f ) ? df
n
??


?	S? (f ) ? df
n
0





Rx (? )
x(t)
Sx ( f )


Impulse response ?(??)



Transfer function ??(??)


Rx (? )
x(t)
Sx ( f )






?	Assuming that the input ??
WSS with ACF


is WSS then the output ??(??) 
will also be
?


Ry (? ) ? Rx (? ) ? ?h (? ) ? ?
??
?


Rx (?


? t ) ? ?h (? )dt


? (t ) ? h(t ) ? h?(?t ) ?


? h(t
??


? t?) ? h?(t?)dt?



?	The PSD Sy(f) is given by the Wiener-Khintchine theorem













Three main sources of noise
?	Thermal noise
\textbullet 	Due to thermal excitation of charge carriers
\textbullet 	Appears as white spectral density
?	Shot noise
\textbullet 	Due to carriers randomly crossing a barrier
\textbullet 	Dependent on DC bias current and is white
?	Flicker noise
\textbullet 	Due to traps in semiconductors
\textbullet 	Has a 1/f spectral density
\textbullet 	Significant in MOS transistors at low frequencies



?	EPT: Every closed physical system at temperature T contains an average 
energy of kT/2 per degree of freedom
?	Consider a gas of electrons having a Maxwellian velocity distribution
? mv 2
p(v ) ?	? e 2kT

where ?? ??	? d?? represents the probability of finding one electron in the velocity 
interval [??, ?? + d??]
?	The average energy of the electron at equilibrium is given by

m	m ?	kT


W ?	m


?	? v 2	?	? ?v 2 ? 
p(v ) ? dv ?


2	2	0	2

?	In 3 dimensions, we have
? 2	2	2	2
v	? vx ? vy ? vz ==>

? 2	? 2	2	3



v	?	v


? 3 ? vx


==> W ? m ?
2


?	kT 	
2




A :	cross section
n :	electron density
J :	current density


U ? R ? I


? R ? A ? J


? R ? A ? n ? q ? v





v	? 1
N




?
i ?1


vi ,	with: N


? n ? A ? L



?	The voltage ?? is then given by:


U ? q ? R


? ?v


? ?u ,	where: u


? q ? R ? v


i		i	i	i 
i	i

?	If no current flows, then < ?? > = 0, < ?? > = 0 and < ?? > = 0.
?	The variances on the other hand will be nonzero


v 2 ? v 2	? 0 and U 2 ? U 2	? 0



?	The ACF of U is given by


R (? ) ? ?R


(? ) ? ? q ? R ? ? ?R


(? )


u	ui	?	L	?	vi
i	?	?	i


where ???????? (??) is the ACF of ??????. ????????
? ?


can be modeled as


R (? ) ? R
i	i


(0)	? e


?0 ,	where ?


is the relaxation 
time and


R (0) ?	2
i


?	If ?????? has a Maxwellian distribution we have:


kT	? q ? R ?2


? q ? R ?2	kT	? ?


R (0) ? v 2	?


and R (? ) ?	? 
R




(? ) ?	?	? e ?0



vi	i	m


ui	?	L	?	vi


?	L	?	m


?	?	?	?
?	The corresponding PSD is given by

? q ? R ?2	kT	2?	? q ? R ?2	kT	1


Su (f ) ? n ?\_A,? L ? ?	?


?	?	0	





? n ? A ? L ? ?	? ?	? 2?0




for f \textbullet 




N	?	L	?


m	1? ?2? f?0 ?


?	L	?	m


2??0


? ? q ? ? ? n?
? ==> R


?	L	? L ?	m
==>



? ? q ??0	?
m	?


? ? A	A 
q2 ? n ??







?	Langevin approach:




R	C
(noisy


R (noiseless)

Vn	Vc



?	Variance of thermal noise voltage across C:

?	df	?	?	1



c	? 1? ?f / f  ?2





2	c	2	2? 
RC


0	c
?	The result also follows directly from the EPT: 
Average stored energy in C:
W ? 1 C ?V 2
2	c

?	Resistor R and capacitor C are in thermal equilibrium and there is only one 
degree of freedom
Mean energy stored in C:



W ? kT


==> 1 CV 2



? 1 kT




==> V 2 ?



2	2	c	2	c






Z(j?)
R1

Vn

RN


Z(j?)


Vn







?	The PSD of noise voltage ???? is given by:



?	The variance of voltage ???? is then given by:





?	The variance of the noise voltage Vn can be obtained without 
computing the integral by using the Bode Theorem stating

where



C?	C0













1 ? lim s ? Z(s)
C?	s??


1 
? 
li
m

s 
? 
Z
(
s
)
C0	s?0



?	Noise PSD of a 1st-order RC circuit




Z(j?)



1	R	R


?R2C


Z( j?) ?
1/ R ?


j?C


? 1? j?RC


?
1? ??RC ?


?	j
1? ??RC ?


R	Vn



Nyquist theorem
==>


Sv (f ) ? 4kT


? ??Z( j 2? f )? ?


4kT ? R
2



n	1? ??RC ?


?	Noise variance of a 1st-order RC circuit from the Nyquist theorem





?
2	(f ) ? df
n


? 4kT




? ???Z ? j 2? f ??? df ?


0	0
?	df	?	?	1	kT


4kT


? R ? ? 1? ?f


/ fc ?


? 4kT


? R ?


2 ? fc


? 4kT


? R ?


?	?
2	2? RC	C



?	Noise variance of a 1st-order RC circuit (by Bode theorem)
C? =C



1 ? lim s ? Z(s) ?



lim


s ? R	? 1



C?	s??


s?? 1? s ? RC	C








C0=?



1 ? lim s ? 
Z(s) ? lim




s ? R	? 0


C0	s?0


s?0 1? s ? RC



?	From the Bode theorem we then have



V 2 ? kT ? 1 ? 1 ? ?

n	?C	C ?
?	?	0 ?



?	Let us consider the following simple mechanical system consisting of a 
proof mass ??, a linear spring with spring constant ?? and a linear 
damper with damping factor ??


?	The deterministic equation of motion of this 
system in the absence of an external force 
is then given by:



d2x  
m ? dt 2


?	b ? dx
dt


? k ? x ? 0



?	According to the dissipation-fluctuation
theorem the damping (loss) is associated with noise
we need to introduce a noise force to model these statistical 
fluctuations according to






?	Using the Equipartition Theorem (1 ? ??



? ?? = ??


) and a derivation


2	??


spring


similar to the one on slide 23 using the equivalent model shown below
shows that the PSD of the noise force ????(??) is independent of 
frequency and given by:

?		In general, to model a noisy mechanical system, the system can 
be modeled by replacing every noisy damper by a noisefree damper 
and a noise force generator alongside the damper



?	To analyze the noise behavior of the system shown below, the rule 
from the previous slide can be applied





Fn1 ?? ? ?
Fn2 ?? ? ?


4kBT
4kBT


? b1
? b2



?	Equivalent model of a noisy resistor (Langevin approach)








R (noisy)


R (noiseless)
R 
(nois
eless
)	In Vn







S	? 4kT ? R
n


? 4kT
n


? G ?


4kT R


(single-sided)	(single-sided)



?	?? is Boltzmann\textbullet s constant (?? = 1.38 \textbullet 10?23 J/K)
?	?? is absolute temperature in K



?	Equivalent model of a noisy forward biased diode (Langevin 
approach)


ID	Rd
noisy		Gd	In 
Vn



S	? 4kT
n


? Rd
2


S	? 4kT
n


? Gd
2


? 2q ? ID


(single-sided)
?	Shot noise


(single-sided)



?	???? is the small-signal conductance of the diode
?	Recall: Large signal current through a pn-diode




?	Equivalent model of a noisy MOST (Langevin approach)




ID

noisy



InD









?	Thermal noise:



GnD


? ?nD



? Gms


? ? nD



? Gm


? 1	WI
? 2


?	? n ? ?


?nD


? ? 2


?	Flicker noise:



?	Total input referred noise:


nD	nD


?	SI
??3



?	Equivalent model of a noisy BJT (Langevin approach)


C

IC

B	noisy	B



E


C



?InC



E







?	Equivalent model of a noisy opamp (Langevin approach)






noisy






?	Requires 3 uncorrelated noise sources
?	All 3 noise sources are needed to have a model independent of the 
generator impedance
?	Current noise sources can be ignored for MOS input stages






?	Noise is a small perturbation compared to ????
?	Analysis can use the small-signal equivalent circuit to which all the noise 
sources are added (Langevin approach)
?	The total output referred noise PSD can be calculated as


Sn,out


(f ) ? ?
k ?1


H (f ) 2 ? S


(f )


where it has been assumed that all the N sources are uncorrelated
?	????(????) are the transfer functions from each noise source k to the output
?	The noise can then be referred to the input by dividing by the square of the 
magnitude of the transfer function from input to output ??(????), according to:
SV	?f ?


S
neq


?f ? ?



nout	
A ?f ? 
2









?Vneq


ACL


? 1? R1
R2





Vout

V	V





Vout


S	2
nR 1


S 2
nR 2


? 4kT
R1
? 4kT
R2




?	Equivalent input noise voltage PSD (assuming no correlation between 
the noise sources)

?	The opamp noise current sources start to dominate at high source 
impedances



?	How to calculate the input referred equivalent noise of an OTA?

Most general configuration	Special case: Ideal voltage
source drive








Vin







These two circuits must result in the same Inout!
noisy


Slide 39	Example \textbullet OTA Noise Analysis (2/5)
?	Langevin model of the OTA
















Small signal equivalent model
?VG3




Inout


S 2
Ink

G


? 4kT

? ?


?Gnk
?G


for: k

? G2 ?


? 1\textbullet 4
?




for: k




? 1\textbullet 4


nk	nDk	mk	mk



W ? L ? f


k	k
? ?	KF
4kT	n





?	Small signal equivalent model
?VG3




Inout



S 2
Ink



? 4kT



?Gnk



for: k



? 1\textbullet 4


G	? ?	?G	? G2


?	?	


for: k


? 1\textbullet 4


nk	nDk	mk	mk


W ? L ? f


I	k	k
? ?	KF



?	Output noise



current:


4kT	n


Gm3


? Gm 4


Inout


? Gm4 ? 
?VG3


?	In2


?	In4 ?	G	G


I	? I


? ==>	I


?	m 4 ? I	? I


?	m 4 ? I	? I


? I	? I	? I	? I


?V	? ?V


? n1	n3	?


nout	G


n1	n 2	G


n3	n 4


n1	n 2	n3	n 4


G3	G4


m
3

m
3
m3


?	Transfer function from ???n to ??nout:


?Vn	noiseless


H(?) ?


Inout
?Vn


? Gm1



?	Calculating ????? :



?Vn


?	?	?	?


Gm
1



?	Concept of equivalent noise resistance:
\textbullet 	For a noisy resistor, we have:
S	? 4kT ? R
nR
\textbullet 	Can express a noise voltage ?Vn as an equivalent noise resistance ??neq:


S	4kT
Vn


?Req



?	For our noisy OTA, this means:


S	? S	? S	? S	? S


? 4kT


? ?G	? G


?	G	? G


? 4kT


?G2 ? R


?Inout	In1


In 2


In 3


In 4


n1	n 2	n3	n 4


m1	nineq








?	Input referred noise resistance in more detail:


Thermal noise: R


? 2? nD1 ? 2? nD3 ?Gm3




? 2? nD1 ? ? ? ? nD3 ? Gm3 ?



nineq-th


G	G2


G	?1	?	G	?


m1	m1


m1	?	nD1	m1 ?


?	? G	? R


? 2?


?	? nD3


? Gm3 ? ==> I




? 4kT ? ?	?G


eq	m1	nineq-th


nD1 ? ?1? ?


G	?	nout


eq	m1


1/f noise:


?
2?	? G	?2	2?


nD1	m1 ?
2?	?	? 
G	?2


W ? L	?	?


R	?	n	? ? m3 ?


?	p	?	n	?1? ? m3 ?


?	1	1 ?   p ?


nineq-f


W ? L ? f


? G	?


W ? L


? f	W


? L ? f ?


? G	?


W ? L	?	?


1	1	m1	3	3	1	1	?	m1	3	3	n ?
?	Assuming M1-M2 in weak inversion and M3-M4 in strong inversion:



? nD1 ?


n1 ,
2



Gm1


? Ib ,
U


? nD3


? 2n3 ,
3



Gm3


?

2
I
b
V


T	P 3


Slide 43	Noise Example \textbullet 1st-Order Low-pass Filter (1/5)

?	Ideal OA (i.e. infinite DC gain and GBW and infinite input resistance) except for 
noise which is modeled by a simple input-referred noise voltage source ??nOA 
(the two current noise sources are neglected assuming the OA is MOS)
?	Output noise voltage for ??in = 0:


V	? Z


? ?I


?	I	? ? H	?V	with	Z	?	R


and	H


? 2 ? s ?c


nout	12


n1	n 2	nOA
nOA	12


1? s ?


nOA


1? s ?


?	Resulting output noise voltage PSD:	c	c


?	with


S
nout


?f ? ?


Z12


?f ? 2 ? ?S


? S n 2 ? ?


HnOA


?f ? 2 ? S


S	? S


? 4kT


and	S


? 4kT ? ?


? ?1? fk ?


I	I	V	?	?


n1


n 2	R


nOA


?	?
m	?
?



?	Substituting into the output noise voltage yields:
? f ?2


8kT ? R 


2 ? ? f ?


?	?	f ?


S	?	?	? c ?


? 4kT


?	? ?1? k ?




Vnout


? f ?2
1?




? f ?2
1?




?	?
m	?	?


? f ?	? f ?
? c ?	? c ?
?	The input-referred noise voltage PSD is then given by:


?	? f ?2 ?


?	?	f ?


SV	? 8kT


? R ? ? 2 ? ?	? ? ? 4kT




?	? ?1? k ?




nin


?	fc ? ?


?	?
m	?	?






?	The variance of the output noise voltage can be calculated by 
integrating the output noise voltage PSD over frequency
?	The contributions of the resistors are 1st-order low-pass filtered. The 
corresponding noise bandwidth is then simply
B	?	?	1	1



n12 ?


2 ? fc


?	?	?
2	2? RC


4RC


?	The variance of the output noise voltage due to ??1 and ??2 is then
simply given by



V 2	?


2 ? 4kT ? Z


?0? 2 ? B


? 8kT


? R2 ? B


? 8kT


? R ?	1


? 2kT



nout


R1,R2	R	12


n12

R


n12


4RC	C


?	The contribution of ??nOA to the output noise voltage PSD is not 
frequency bounded and hence its contribution to the output noise 
voltage variance is (in theory) infinite
?		In order to evaluate the output noise voltage variance, we need 
to account for the amplifier finite bandwidth


?	If the amplifier transfer function is assumed to be ??(??), then the 
transfer function ??nOA becomes:


H	?s ? ?


2 ? s ??	


nOA


2	?	1	?


1? A ?s ? ? ?1? A ?s ? ? ? s ??
?	?
?	With ?? = ????
?	Note that for ?? ? ? , we recover the earlier result
?	If ?? is simply approximated by ?? = ??0/?? where ??0 = ????/???? is the gain- 
bandwidth product with ???? being the compensation capacitance, the 
transfer function becomes


H	?s ? ?


2 ? s ??	


nOA


?	? 2	?


s2 ?	? ?	? ? ? ? s ? 1

?0	? ?0	?





?	It can be shown that the noise bandwidth corresponding to the 2nd- 
order transfer function ??nOA is given by
??	??


B	?	1


?	H	?f ? 
2 ? df ?	1 ?	H




?? ? 2 ? d?


? ?0


? 4 ? ?0 ??


? ?0


for ?


?? \textbullet 1


nOA


HnO
A


?0? ?


nOA


8?	?


nOA


16	2 ? ?0 ??	16



?	The output noise voltage variance due to ??nOA is then given by



V 2	? 4kT ? B	? R


? 4kT


? ?0


? ?  


? kT ? Gm


? ?  


? ? kT


nout Vn 3


nOA	nOA



16 G



4C	G
4C


m	c	m	c

?	The total output noise voltage variance then becomes

?	Since the voltage gain in the passband is unity, this also corresponds 
to the total input-referred noise voltage variance

























Fundamentals of LNA design for CT sensor interface
CT LNA\textbullet S FOR RESISTIVE AND 
CAPACITIVE SENSORS



?nG


?1	WI saturation
? ?
SI saturation
? 3


G	? ?2  ? ? qs  ? ?CGS  ? CGD ?

Gi	2n



?qs


?	1	L2


?	? ?UT
? 4	L2


WI saturation


?	?
?15


? ? ?VP


?Vs ?


SI saturation





?	Drain noise current:


V ? VG ?VT0
P	n








?	Gate noise current:
?	??????????????????   is correlated w/ ??????????????D and most important at RF





?	For an amplifier in saturation it makes sense to refer the drain noise to 
the gate










where ??????????  is given by


?	1 ? n 



WI and saturation


RnG


? G	nD
G2


? ? nD
G


?	2 Gm
? 2	n	1


m	m	?  ?	?



SI and saturation



??3  Gm	Gm



?	Two main sources of Flicker noise
\textbullet 	Carrier number fluctuation ????N (Mc Worther model)
\textbullet 	Carrier mobility fluctuations ?????? (Hooge model)
?	The PSD of the gate referred voltage fluctuations is:




where


S?V 2


?f ?	?


K?N
2


and


S?V 2


?f ?	?


K??




nG	?N


W ? L ?Cox ? f


nG	??


W ? L ?Cox ? f




?	Inversely proportional to frequency and to gate area
?	Note that ??????N  and ????????  are slightly bias dependent





S?V 2


?f ? ? S?V 2


?f ?


?	S	2


?f ?


nG	nG	?N	nG	??

?	Usually carrier number fluctuations dominate over mobility fluctuations
?	For design purposes, the gate referred noise PSD can be considered 
as bias independent





?	PSD\textbullet s of noise currents ??????C  and ??????B:
I?F
SI	?f ? ? 2q ? IC ,	SI	?f ? ? 2q ? IB ? Kf ? B  



nC	nB


m
shot noise


\_,f
Flicker noise


?	Thermal noise of resistances ????, ???? and ????:



S
nRB


? 4kT ,
R


S
nRC


? 4kT ,
R


S
nRB


? 4kT
R


B	C	E







?	Base current is already at the base:




S
n ,eq


? S	?
nB


2q ? IB




?	Collector current needs to be referred back to base:




S
S	?	nC  ? S




? 2q ? Ic


?	4kT ? R


? 4kT ? ?	1


? R ?


Vn ,eq	G2


VnR	G2


B	? 2 ? G	B ?


m	B	m	?	m	?





Assumptions:
\textbullet 	??????,B????????????  is sufficiently 
small
\textbullet 	??????????????????????????  is negligible



S
nM1



S
nM2


? 4kT

? 4kT


?	1
Gm1
?	1
Gm2




?	Total equivalent input referred voltage noise:





Assumptions:
\textbullet 	??????,B????????????M????/????  are 
sufficiently small
\textbullet 	??V????????????????????M????/????  are 
negligible



S
nM1/2/3


? 4kT ?	1
G




?	Total equivalent input referred voltage noise:


m1/2/3









sufficiently

negligible



S
nM1/2/3


? 4kT ?	1
G


m1/2/3









?	Total equivalent input referred voltage noise:
























M1 and M2 are assumed 
to be identical!


?	Noise from distinct devices is uncorrelated








Total output referred current noise



S
nout,tot


? S
nM1


?	S
nM2


?	S
nM3


?	S
nM4






S	? 4kT
n


?Gm




Referring it back to the input:



SVneq    ?


1
2
m1/2


? S
nout,tot




?	Total input referred noise:

1	1	?	G	?


S
neq


? G2


?2 ? 
SInM1/
2


? 2 ? S
nM3/4


? ? 2 ? 
4kT


?	? ?1?	m3/4 ?
G	G


m1/2



\textbullet 
\_\textbullet 
m,1/2	?
S
neqM1/
2


m1/2 ?








?	Voltage drive:
???? ? ????
?	Current drive:
???? ? ?






S	? R2 ? S


?	Noise figure:


NF ?


s	a ? 1?   a ? 
1? NF


? 1?


Vn,eq


s	In,eq


N	N	ex



4kTR



?	Voltage drive: NF????????  ?? ????/????
?	Current drive: NF????????  ?? ????


s	s	s





RB ? 10 ?
? ? 100











?	For large source resistances ???? the input referred current noise 
becomes dominant
?	Optimum ???? exists for a given equivalent voltage and current noise









For sufficiently large A:

vx ? 0


vout
vx


? ? Cin
CFB









?	Question:
How to choose W and ??B????????????  for an optimum SNR?






?	Circuit analysis:
Compute ?????????? for the 2 circuits and equate them:




? C	? C	? C	?2
S	? ?   FB	GS	in ? ? S


vneq


C ?G


ind


?	in	m
?Gms



?
WI and saturation


S	? 4kT ? ? 2


inD


?
?  ?G



SI and saturation


??3	ms


G	? n ?G ,


V ? VG ?VT0



ms	m	P	n




2	G	? ? ?C


? ? W


??V


?V ? ?	?


2IIBIAS




C	?	?C




?W ? L


m	ox   
? L ?	P	S


n ?V ?V ?


GS	3	ox


\textbullet \textbullet \_\textbullet ?
?


\textbullet ,?	P	S











Simulation parameters:



L ? Lmin


? 180 nm,


tox


? Lmin


/ 50


? ? 400 cm2 / ?Vs?,


n ? 1.4


CFB


? 100 fF,


Cin


? 500 fF




Optimum transistor width:


?	Total input referred voltage noise:

? C	? k? ?W ? C	?2


S
neq


? 4kT ? n ? ? FB
i
n ? ?	?


?	in	?




Simulation parameters:



L ? Lmin


? 180 nm,


tox


? Lmin


/ 50


? ? 400 cm2 / ?Vs?,


n ? 1.4


CFB


? 100 fF,


Cin


? 500 fF



Optimum transistor width:

?	Total input referred voltage noise:

? C	? k? ?W ? C	?2	1	L	1
Sv	? 4kT ? n ? ?   FB	in ? ?	?	?


neq


C	? ?C	W	?V ?V ?


?	in	?	ox	P	S




?	Sensor:
A device which converts information from one energy domain into 
the electrical domain
?	In the electrical domain, (analog) signal conditioning, digitization 
and finally digital signal processing (DSP) and storage can take 
place






?	Most sensors provide analog information at their outputs
?	Analog-to-digital (A/D) conversion is required
to benefit from DSP and convenient data displaying 
and storage
?	There will be a lecture on A/D conversion for senor 
interfaces towards the end of the course



?	Transduction:
Converting a signal from one energy domain into another
?	Both sensors and actuators are transducers
?	Typical signal/energy domains:
\textbullet 	Magnetic (Mag)
\textbullet 	Mechanical (Mech)
\textbullet 	Thermal (Therm)
\textbullet 	Optical (Opt)
\textbullet 	Chemical (Chem) and
\textbullet 	Electrical (Elec)
?	Sources of uncertainty (error):
\textbullet 	Sensitivity to unintended electrical and nonelectrical quantities
\textbullet 	Noise (thermal, 1/f, \textbullet )
\textbullet 	Electromagnetic interference (EMI) \textbullet 



?	If we write the different signal domains (i.e. energy forms) as a column 
vector, the transduction action of a sensor can be written as a matrix 
with certain sensitivities acting on an input vector (sensor excitation) 
and producing an output vector (sensor output signal).



? E	?


? ?mag,mag	?mag,mech	?mag,therm	?mag,opt	?mag,chem	?mag,elec ? ? E


?	? E	?


mag


? ?	?	?	?	?	?


? ?	mag ?	?	mag,os


? Emech ?


?	mech,mag	mech,mech	mech,therm	mech,opt	mech,chem	mech,elec ?


? Emech ?


? Emech,os ?


? E	?


?
therm,mag


?therm,mech


?therm,therm


?therm,opt


?therm,chem


?
therm,elec	E


?	? E	?


?	therm ? ? ?	? ? ?	therm ? ? ?


therm,os ?


? Eopt   ?	? ?opt,mag	?opt,mech	?opt,therm	?opt,opt	?opt,chem	?opt,elec   ? ? Eopt   ?


? Eopt,os   ?


? E	?


?
?	?	?	?	?	?


? ? E


?	? E	?


chem
?	?


?	chem,mag	chem,mech	chem,therm	chem,opt	chem,chem	chem,elec ?
?


chem
?	?


chem,os
?


? Eelec ?


? ?	?	?	?	?	?	? ? Eelec


?	? Eelec,os 
?


?	elec,mag	elec,mech	elec,therm	elec,opt	elec,chem	elec,elec ?

?	The additional additive vector models offsets 
which are often present in sensors


Slide 6	Example: Transduction from Emech to Emech
?	Blue entries in the transduction matrix represent non-zero coefficients



? E	?


? ?mag,mag


?mag,mech


?mag,therm


?mag,opt


?mag,chem


?mag,elec ? 
? E


?	? E	?


mag	? ?	?


?	?	?	?


? ?	mag ?	?


mag,os


? Emech ?


?	mech,mag


mech,mech


mech,therm	mech,opt	mech,chem	mech,elec ?


? Emech ?


? Emech,os ?


? E	?


?
therm,mag


?therm,mech


?therm,therm


?therm,opt


?therm,chem


?
therm,elec	E


?	? E	?


?	therm ? ? ?	? ? ?	therm ? ? ?


therm,os ?


? Eopt   ?	? ?opt,mag	?opt,mech	?opt,therm	?opt,opt	?opt,chem	?opt,elec   ? ? Eopt   ?


? Eopt,os   ?


? E	?


?
?	?	?	?	?	?


? ? E


?	? E	?


chem
?	?


?	chem,mag	chem,mech	chem,therm	chem,opt	chem,chem	chem,elec ?
?


chem
?	?


chem,os
?


? Eelec ?


? ?	?	?	?	?	?	? ? Eelec


?	? Eelec,os ?


?	elec,mag	elec,mech	elec,therm	elec,opt	elec,chem	elec,elec ?

?	A simple spring with a proof mass attached 
to a stationary mass can detect 
accelerations as displacement
mechanical energy (kinetic energy)
is transformed into mechanical energy 
(spring potential energy)
?	A mechanical offset energy can exist 
because the spring can e.g. be prestressed



?	The transduction inside a sensing element can also take place via 
multiple transduction steps. Such a transducer is called a tandem 
transducer
?	Example: an absorptive coating on a bimetal strip
\textbullet 	In the 1st transduction step, optical energy (photons) is converted into 
thermal energy
\textbullet 	In the 2nd transduction step, the thermal energy deflects the bimetallic strip, 
that is the optical energy is transduced into mechanical energy













?	A tandem transducer can be modeled using two transduction matrices, 
representing the two successive transduction steps
?	Note: Tandem transduction requires an intermediate non-electrical 
energy domain
?	Example:
\textbullet 	In the transducer on the right, acceleration 
(mechanical energy) is transduced into 
displacement (mechanical energy)
\textbullet 	To readout the displacement signal 
one typically converts the displacement 
into a change in capacitance which can 
be processed by the readout 
electronics, i.e. there is a mechanical 
to electrical energy transduction
Since there is no intermediate transduction step
going to a non-electrical energy domain, the accelerometer on this slide is
not a tandem transducer



?	Consider a transduction to the electrical domain represented by the 
following transduction matrix


? E	?


? ?mag,mag	?mag,mech	?mag,therm	?mag,opt	?mag,chem	?mag,elec ? ? E


?	? E	?


mag


? ?	?	?	?	?	?


? ?	mag ?	?	mag,os


? Emech ?


?	mech,mag	mech,mech	mech,therm	mech,opt	mech,chem	mech,elec ?


? Emech ?


? Emech,os ?


? E	?


?
therm,mag


?therm,mech


?therm,therm


?therm,opt


?therm,chem


?
therm,elec	E


?	? E	?


?	therm ? ? ?	? ? ?	therm ? ? ?


therm,os ?


? Eopt   ?	? ?opt,mag	?opt,mech	?opt,therm	?opt,opt	?opt,chem	?opt,elec   ? ? Eopt   ?


? Eopt,os   ?


? E	?


?
?	?	?	?	?	?


? ? E


?	? E	?


chem
?	?


?	chem,mag	chem,mech	chem,therm	chem,opt	chem,chem	chem,elec ?
?


chem
?	?


chem,os
?


? Eelec ?


? ?	?


?	?	?	?


? ? 
Eelec


?	? Eelec,os ?


?	elec,mag	elec,mech


elec,therm	elec,opt	elec,chem


elec,elec   ?



?	There is one (desired) sensitivity plus 2 cross-sensitivities and an offset
?	This could be the transduction matrix of a photodiode
\textbullet 	Desired sensitivity: ??elec,opt
\textbullet 	Cross-sensitivities: ??elec,elec (e.g. electromagnetic 
interference), ??elec,therm(diode current strongly 
depends on temperature through ???? = ????/??
\textbullet 	Offset due to dark current



?	Active sensors:
Require an external source of excitation 
Examples:
\textbullet 	Resistor-based sensors (AMRs, GMRs, thermistors, \textbullet )
?	Passive or self-generating sensors:
Generate their own electrical output signal without requiring external 
voltages or currents
Examples:
\textbullet 	Thermocouples (thermoelectric voltage generation) and photodiodes 
(photocurrent)
?	Modulating sensors:
Measure the desired quantity by modulating
?	Analog vs. Digital sensors:
\textbullet 	Analog sensors provide signals which are continuous in time and 
amplitude
\textbullet 	Digital sensors provide discretized outputs



?	Deflection mode sensors:
\textbullet 	The sensor response to an input is a \textbullet deflection\textbullet 
or more general a deviation from its equilibrium 
position in the absence of an input. For an ideal 
sensor, this deviation is proportional to the 
measurand of interest
?	Null mode sensors:
\textbullet 	The sensors or instruments exert an influence on 
the measured system opposing the effect of the 
measurand. Ideally, the influence of the measurand 
and that by the sensor/instrument are precisely 
balanced in magnitude but opposite in sign, 
resulting in a null measurement. This nulling is 
typically achieved via feedback.
?	Null mode instruments are very linear and can produce very accurate 
measurements, however, they are not as fast as deflection instruments 
(which are open loop systems)



?	Self-generating sensors require no external power supply
?	Due to the absence of an external source, there is no additional source 
of uncertainty/error
?	A self-generating sensor draws all its required energy from the 
measurand (source loading)
?	Example: Potentiometer for angle measurements




?	Use an additional external source whose energy is modulated to 
produce the output signal
?	The additional source introduces an additional source of error
?	A modulating sensor takes less energy from the measurand
?	Example: Non-contact displacement measurement












The code tracks on the rotating disc 
allow for a unique (digital) encoding of 
the current shaft angle





Illustrations courtesy TU Delft



?	The resistance of a slab of conductive material can be expressed as:



where ?? is the bulk resistivity of the material
?	Under mechanical stress (strain) the change
in geometry produces an associated resistance 
change


?R ? L0


? ?? ?


?0 ? ?L ?


?0 ? L0


? ?A ==>


2
0	0	0


?R ? ??
R0	?0


? ?L
L0


? ?A
A0


?	Assuming that the volume stays constant
the expression can be simplified according to


Slide 15	Strain gauges with applied force (1/2)
?	From previous slide:



?R ? ??
R0	?0


? ?L
L0


? ?A ,
A0


?
A

?

?

?
L

A
0

L
0











?	For metal films the change in resistivity can be 
modeled according to:


?L ? 0 ==> ? ?R ?	? 0


?
? R0


?
?strain






?	Bringing the 2 results together:



k: gauge factor


Slide 16	Strain gauges with applied force (2/2)
?	From previous slide:










k: gauge factor
?	For tensile stress we have



?L ? 0 ==> ? ?R ?	? 0





?	For compressive stress the opposite is true


?
? R0


?
?strain







?	The simple strain gauge structure on this and the previous slides is 
constructed to give a large ??? in the
up/down direction and a small ??? in 
the left/right direction!

?	Ideally we have for the force applied 
on this slide:
Applying a force in the insensitive direction:


?	For commercially available sensors 
this is typically satisfied
?	When designing your own sensors you 
need to ensure this by a proper 
geometry selection



?	A simple resistive divider can be used to read 
out resistive sensors
?	Features of a half-bridge:
\textbullet 	Simple structure
\textbullet 	The relationship between ??? and ??out is non-linear
\textbullet 	For relatively small values of ??? the circuit tends to produce small 
changes in the output voltage with a large offset	difficult to measure



?	We can also make the bottom resistor the 
sensing element

?	The features are virtually identical to the other 
half bridge configuration (but: sign reversal!)



?	The offset present in the half-bridge can be 
removed using the differential structure 
shown on the right
?	Features of the single-point full bridge
\textbullet 	No intrinsic offset
\textbullet 	Output does not change linearly with the change in resistance ???
?	Balance condition:




?	Bridge configurations can be made more 
sensitive by incorporating two sensing 
elements

?	Features:
\textbullet 	No intrinsic offset, non-linear relation between ??? and ??diff
\textbullet 	Two times better sensitivity than 1-point full-bridge
?	Balance condition:




?	The nonlinearity problem can be solved if
a differential measurement becomes possible 
(i.e. one sensing element produces
+??? resistance change and a second 
one produces ????)

?	Features:
\textbullet 	No offset and linear relation between ??? and ??diff
?	Balance condition:



?	If differential sensing elements are 
available we can also construct a full 
bridge where all sensing elements 
are varying


v ? 1 ? ?1?




?R ? ?V



1	2 ?	R ?	bias
?	?
1 ?	?R ?


v2 ?


2 ? ?1?


R ? 
?Vbias


?	?
?	Features:
\textbullet 	No offset, intrinsically linear, highest sensitivity
\textbullet 	Requires a differential sensing element
?	Balance condition:







?	Sensitivity:
The bridge sensitivity ?? specifies the output voltage when the bridge is 
excited with 1 Volt and the sensor is at full scale


Example: 1-point full-bridge with ???
?? ????


= 20 %


vdiff
Vbias


? 1 ?	?R	?
2 2R ? 
?R


1 ? ?R R	
2 2 ? ?R 
R



S ? 
45 
mV
/V



?	Accuracy:
The bridge accuracy ?? measures the deviation of the bridge from an 
ideal bridge


v	? v


? ?R


? 0 ? ?


? ?R



diff,lin	diff ? R	?	R


?	?

v	? ?R ? ? v






?R ?0
R
? ?R ?



diff ? R ?	diff,lin ? R ?
A ?	?	?	?	?

v	? ?R ?

diff ? R ?
?	?



?	Bridge resistance:
The bridge resistance is the resistance of the bridge measured 
unloaded across the signal terminals
Note: The bridge resistance determines the static power consumption 
of the bridge!
?	Offset error:
The bridge offset is the bridge output voltage which is produced when 
the measurement parameter is zero
?	Drift:
The drift of a bridge is the change in bridge output voltage with 
environmental condition over time. Here, typically temperature is the 
most critical source for drift





Readout w/ discrete 
electronics





On-chip readout 
electronics




?	Signal conditioning takes care of:
\textbullet 	Reading out the bridge circuit w/o significant loading effects
\textbullet 	Amplification
\textbullet 	Anti-aliasing filtering
\textbullet 	Differential-to-single-ended conversion for discrete readout electronics





?	Differential signal:


?	Common-mode signal:








?	The differential signal is most often much smaller than the common-


mode signal

?	Example: ??bias



= 1 ??, ???
??




= 1 %



vdiff


? ?R ?V
R	bias
V



? 10 mV




50x


vCM


?   bias
2


? 0.5 V



?	The information carrying signal ??inDM is 
amplified by the differential gain ??DM

?	The undesired common-mode signal
??inCM is amplified by the common 
mode gain ??CM


?	In practice, ??CM is non-zero and the output depends on both ??inDM and
??inCM according to:
?	Ideally, we would like ??CM to be zero or at least much smaller than ??DM



?	The common-mode rejection ratio (CMRR) characterizes the ability of 
an amplifier to suppress common mode signals at its input w.r.t. 
differential signals (i.e. to \textbullet distinguish\textbullet DM from CM signals)
?	From previous slide:

?	The CMRR is then defined as:

?	Note: the CMRR is most often expressed in dB!
?	Example: If an amplifier has a CMRR of 60 dB and there is a 
differential signal of ???inDM = 0.5 mV and a common mode signal of 
???inCM= 0.5 V at its input, the changes in output voltage due to ???inDM 
and ???inCM are equal!



?	The simple opamp based difference amplifier can be used to amplify 
voltage differences at its input











?	Let\textbullet s calculate ??DM, ACM and the CMRR!





?	1st assume ??


= ?, ??2 = ??? = 
?? ? ??1 = ?? ? ???


2	1


?	2nd assume ?? ??	= ?, ??2 = ?? ? ??1,


??? =


? ??? , 
??? = 
??1 + 
??????


2	1	1
?	Note: ??DM, ??CM and CMRR are independent of ??? and scale positively 
with ???. Numerical Example: ??? = 1 % , ?? = 20






?	3rd assume ??


= ??DC


/(1 + ?? ? ??DC )
GBW




?	Note: The finite opamp gain and GBW affect ??DM and ??CM in the same 
way and therefore, the CMRR remains unaltered!
?	Apart from the finite CMRR due to resistor mismatch, the difference 
amplifier suffers from source loading due to the finite value of ??1 (this is 
particularly true for IC realizations where large resistor values require 
large areas)	we will address these problems by introducing so- 
called instrumentation amplifiers later in the course



?	A bridge structure can also be used to measure reactive sensors (i.e. 
capacitive and inductive sensors)
\textbullet 	Replace the resistors by generalized impedances
\textbullet 	Replace the DC excitation by an AC source
?	The balance condition is:



?	This complex balance equation can be 
written as 2 real valued equations:



Z1 ? Z2 ?
? 


R1 ?


jX1


? R2 ?


jX2


Z4	Z3


R4 ?


jX4


R3 ?


jX3



?	For the capacitive bridge according to Wien, the balance condition is:

R1	?	R2



R4 ?


j ? ?


?
? ?C ?


Rx ?


j ? ?


?
? ?C ?


?	4 ?	?	x ?






?	For the inductive bridge according to Maxwell, the balance condition is:



R ? ? 1


? ?	R3 ? R1



4	?	j?C ?




R ? j?L


?	?	x	x




Integrated Interface Circuits \textbullet Chapter 2: Sensor types



?	The virtual short of an opamp can be used to enforce a zero bridge 
voltage. Note: The balance conditions from the previous slides are not 
satisfied! Why?
?	For a resistive bridge with ??1 = ??2 = ??4 = ?? and Z3 = ????
we obtain for the required feedback
voltage ????
ileft ? iright ==>


Vex


? ?Vex


?	vx ?


2 ? R	R ? Rx

?	Setting as usual ???? = ?? + ???, we find




?	All bridge configurations are typically read out with a low offset, high 
input impedance differential amplifier. The amplifier is then typically 
followed by an LPF (anti-aliasing filter) and an ADC






























MAGNETIC FIELD SENSORS



?	Magnetic field sensors are mostly distinguished by their dynamic range 
and frequency range (i.e. bandwidth)


Min. B field
Max. B field
Frequency range
Induction coils
100 fT
unlimited
0.1 mHz \textbullet 1MHz
Hall sensors
10 nT
20 T
0 \textbullet 100 MHz
Magnetoresistive 
sensors
100 pT
100 mT
0 \textbullet 100 MHz
Fluxgates
10 pT
1 mT
0 \textbullet 100 MHz
SQUIDs
5 fT
1 \textbullet T
0 \textbullet 100 kHz
?	Therefore, different applications clearly require different magnetic field 
sensors
?	In this course we will focus on Hall sensors because:
\textbullet 	they can nicely be integrated in CMOS technologies
\textbullet 	They have an enormous market share (Asahi Kasei Co. sells over 100 
million Hall elements per month)



?	As ingredients, we need the Lorentz force (which in general relates 
the electromagnetic world (Maxwell equations) to the mechanical world 
(Newton\textbullet s laws of motion)
?	Next, we need to introduce scattering into Newton\textbullet s law of motion




where


???


is the drift 
velocity and 
?? the mean 
free time 
collisions


?	Combining these two equations yields:
?	Assumptions: We apply an electric field in the ???? direction with a 
resulting current flow and no current flows in the y- and z-directions:





?	On the previous slide we found:


?	Writing this equation in component form




?	And substituting:
?	Yields:




?	From the previous slide:




?	Assume that according to the drawing on this slide ???? ? 0, ???? = 0, then 
the Hall voltage ??hall becomes:


Vhall


d /2 
?	?
?d /2 


Ey dy


? vx ? Bz ? d


? ? ? Ex ? Bz ? d


? ? ? ? ? Jx ? Bz ? d ?


? ? ?
h


? I ? Bz






?	Recall: The resistivity is given by:


? ? 1
?


?	1	
e ? n ? ?


? Jx ? h ? d





?	Note: To maximize ??hall, we need minimum ?, maximum ?? and low 
doping (small ??), also ??hall is independent of the sensor surface ?? ? ??



?	The effectively measurable Hall voltage depends on the geometry of 
the specific Hall device. This is modeled by the geometry factor ??




?	The most important noise sources for Hall devices are:
\textbullet 	Thermal noise (eventually determining the limit of detection (LOD))
\textbullet 	1/f noise
\textbullet 	Noise of the signal conditioning electronics, EMI
?	The intrinsic thermal noise of a Hall device originates in the resistance 
between the two voltage terminals of the Hall device
?	Noise power:


?	Signal Power:


?	Resolution:
?	Example (Si):
?? = 1016 cm?3, ? = 2 \textbullet m, ?? = 1 ?cm, ?? = 300 K, ?? = 15 k?, ????? =


0 V 
A?T


, ?? = 0.5 mA



?	From the previous slide:
?	To achieve a good resolution we need:
\textbullet 	materials with a high mobility (large ??)
\textbullet 	low doping levels (small ??)
\textbullet 	thin devices (small ?)
\textbullet 	large currents ??
?	BUT: ??, n and ? are limited by
\textbullet 	heating
\textbullet 	available power (e.g. in battery operated devices)
\textbullet 	Carrier velocity saturation






Stress 
(e.g. due to 
packaging)









?	In the absence of a magnetic field ?? = 0, the Hall plate can be 
modeled as a simple Wheatstone bridge
?	Non-uniform stress inside Hall devices leads to intrinsic offsets in this 
Wheatstone bridge
?	This offset is superimposed on the actual Hall voltage resulting in an 
(unknown) offset
?	Typical (equivalent) offset values are between 5 and 30 mT





?	According to the above figure, we can couple two Hall sensors, which 
are rotated by 90\textbullet (i.e. the bias current port and the voltage sense port 
are swapped), to cancel the intrinsic offset
?	Key idea:
The offset voltages due to doping gradients, temperature gradients and 
mechanical stress gradients are 180\textbullet out of phase while the generated 
Hall voltages are in phase







? : Hooge parameter (1/f noise model) 
GF: geometry parameter
Rsheet : sheet resistance
n: doping concentration





?	A typical (asymptotic) noise PSD of a Hall sensor is shown above
?	The lower frequency noise is dominated by the sensor\textbullet s 1/f-noise


?	The white noise part of the spectrum is determined by the thermal 
noise associated with the loss of the Hall element
?	Typical corner frequencies are in the range of 1 to 100 kHz



?	Offset and 1/f noise can be greatly reduced by the so-called spinning 
current or switched Hall technique
?	Intuitive explanation: 
Assume ??? > 0

furthermore

















Slide 56


CL
K 
ph
as
e









?	The spinning current method implemented above produces an output voltage 
with a DC component equal to the offset and a frequency component 
proportional to the Hall voltage at the clock frequency (Fourier expansion of a 
square wave!)
?	How is this useful? Answer:
The amplifier in the circuit above also possesses an offset and 1/f-noise. By 
\textbullet modulating\textbullet the information to the clock frequency we can greatly mitigate the 
effects of both noise and offset











?	A complete \textbullet switched Hall system\textbullet consists of
\textbullet 	The switched Hall plate together with an amplifier
\textbullet 	The demodulation switches to translate the signal from the switching 
frequency back to DC
\textbullet 	A final low-pass stage to remove the signal around the 2nd harmonic



Conventional Hall readout	Low-power Hall readout

?	In a conventional implementation of the spinning current technique, the Hall 
element and the amplifier are biased with separate dedicated bias currents

?	In the low-power configuration, the bias current for the Hall element is reused 
to bias the amplifier





Source: Kejik et al., \textbullet Integrated Hall Microsystem with current re-use\textbullet , Sensor Letters, vol. 5, 1-3, 2007

?	Spinning current technique applied to 4 Hall elements
?	Demodulation is achieved by commuting the input differential pair in 
the switching box
?	Amplification is properly defined by feedback resistors feeding the 
output signal back to a dedicated feedback differential pair



?	Technology: 0.8 \textbullet m standard CMOS 
technology
?	Total current consumption of 2.4 mA
\textbullet 	80 % for Hall sensor and electronics
\textbullet 	20 % for remaining electronics
?	DDA configuration:
\textbullet 	Simple demodulation implementation
\textbullet 	Small silicon area
?	Spinning current technique:
\textbullet 	1/f noise suppression
?	Features:
\textbullet 	Low-power, small size
\textbullet 	High performance (resolution: 500 nT )
Source: Kejik et al., \textbullet Integrated Hall Microsystem with current re-use\textbullet , Sensor Letters, vol. 5, 1-3, 2007



?	The so-called \textbullet Seebeck effect\textbullet refers to the 
conversion of temperature difference into an 
electromotive force (emf)

?	Any emf ??emf modifies ohm\textbullet s law according to:


J ? ?


? ???V


?	Eemf ?


where ?? is the electric potential and ?? is 
the electric conductivity

?	The Seebeck effect produces an emf ??emf
according to:


Eemf


? ?S(T ) ? 
?T


Sou
rce: 
http
://e
n.wi
kip
edi
a.or
g/wi
ki/F
ile: 
The
rmo
elec
tric
\_Ge
ner
ator
\_Di
agr


where ??(??) is the Seebeck coefficient


am.svg


(material property) and ??? is the temperature gradient
?	For \textbullet standard\textbullet materials, the Seebeck coefficient takes on values from
? 100 ?V/K to 1 mV/K



?	Thermocouples make use of the Seebeck effect to measure the 
temperature difference between two objects

?	For ??? = 0, the emf induced by the Seebeck effect can be directly
measured as a voltage difference
?	The measured voltage can be calculated by integrating (\textbullet summing up\textbullet ) 
all emfs due to material differences inside the sensor shown below




?	From the previous slide:
?	Clearly, to maximize the emf, one should use materials with very 
distinct Seebeck coefficients (e.g. chromel and alumel)
?	Note: To compute ??sense one does not need to invert the above 
equation for every measurement
?	Instead, we use the following \textbullet trick\textbullet :

Define:





?	From previous slide:
?	Typically, ??standard is chosen as 0 \textbullet C
?	The so-called characteristic function of the thermocouple ??(??) is 
provided by thermocouple manufacturers and metrology standards 
organization over a range of temperatures for different specific 
thermocouple types
?	Note: To extract the temperature ??sense from the above equation, the 
reference temperature needs to be known
typically evaluated by two different methods
\textbullet 	\textbullet Ice bath\textbullet method: Immerse the reference junction in a semi-frozen bath of 
distilled water at atmospheric pressure ? ??ref = 0 \textbullet C (not so practical)
\textbullet 	Reference junction thermometer (thermocouples are still useful, because 
their leads can be exposed to extreme environmental conditions)



?	The table lists some commonly used materials for thermocouples:

material stack
sensitivity (at 0\textbullet C) 
[\textbullet V/K]
range [\textbullet C]
iron/constantan
45
0..760
copper/constantan
35
-100..370
chromel/alumel
40
0..1260
platinum/Pt+Rd
5
0..1500
?		Thermocouples give quite small output voltages, however, they 
are intrinsically offset free because they always measure a 
temperature difference which results in ??emf = 0 for ??? = 0 K
?	The small output voltage is problematic due to the non-zero offset of 
real-world voltage amplifiers
?	Solution idea:
We could increase the emf, by connecting many thermocouples in 
series \textbullet 



?	According to the previous slide, one can increase the output voltage of 
a thermocouple by connecting many of them in series resulting in a so- 
called thermopile









?	While the problems associated with the small output voltage are 
solved, a new problem arises:
\textbullet 	There is increased thermal conduction between cold and hot parts 
increased source loading and more heat flux required to build up
a temperature difference



?	Transistors are natural 
temperature sensors
?	However, manufacturing 
tolerances cause temperature 
errors of up to 3\textbullet C





?	Key idea:
Producing a temperature stable reference which cancels a positive 
temperature coefficient with a negative one



with


VREF 
?T ? 
? 
VCTA
T ?T 
? ? 
K 
?VPTA
T ?T 
?


??PTAT ??	a voltage which is proportional to absolute temperature
??CTAT(??) a voltage which is complimentary to absolute temperature


?? a temperature independent constant which makes ??REF
temperature
?	This type of reference is known as bandgap reference


independent of





?	Collector current of a BJT as a function of its base emitter voltage and 
emitter area for ??BC = 0 (diode connected device)


?	VEB


? VEB \textbullet kT q


VEB


I	? I? ? A


? ?ekT q


? 1?	?	I? ? A


? ekT q


C VBC ?0	S	E	?	?	S	E

?	Conversely, the base emitter voltage can be expressed as a function of 
the collector current


VEB


? kT
q


? ln ?	IC
I? ? A


?	kT
? 1? ?	q


? ln ?	IC	?
I? ? A


? S	E	?	? S	E ?



?	A PTAT voltage can be generated using 2 
diode connected BJTs



VPTAT


? ?VEB


? VEB1 ?VEB2


? kT


? ln ?	Iq


? ? kT


? ln ?	Iq	?



q	?I? ? A	?	q	?I? ? A	?
? S	E1 ?	? S	E 2 ?


? kT ? ln ?Iq  ? IS?




? AE 2 ? ? kT


? ln ? AE 2 ?



q	? I? ? A


? I ?	q


? A	?


?? S	E1


q ??


?	E1 ?



?	Numerical example:
??E2 = 10 ? ??E1

?	1.381?10?23 J/K	199 ?V



VEB ?


1.6 ?10?19 C


? ln?10? ?T ?


K ?T



?	Note: A true CTAT voltage does not exist!
?	Best approximation: Start from the current through a pn diode:
VEB


I	? I


? ekT q


? I? ? e


, E : 
bandga
p 
energy, 
E


? 1.13 eV 
(Si)


D	S	S	g	g


I? ? K



?T ? ,


? ? 3, I


? I?


?Eg
? e kT


Eg \textbullet kT
==>  I? \textbullet I ,


I? ? I


S	1	S	S	s	s	s	D


kT	? I? ?	? K


?T ? ?


VEB


? Eg / q ?	? ln? s ? ?? VG



? UT ? ln?	1	?


m	mq


? ID ?	?	ID	?


V
G	UT



VG ? VG0


?	a ?T


(good from 
200 K to


400 K)





?	The terms ????


and ? ????
??


provide the desired 
CTAT


voltage while the temperature dependence in the 
argument of the natural logarithm is undesired
results in curvature of the CTAT voltage and 
therefore in an error in the \textbullet temperature independent
curvature correction (not covered



?	Basic structures:





















?	Series form: ??


= ??


? ??


+ ??


= ??2 ? ??



+ ??


REF


PTAT	2


CTAT


??1


PTAT


CTAT





?	Parallel form: ??	=


? ??	= ??3 ? ??



+ ??3



? ??


REF


3	??1


PTAT


??2


CTAT


?	To achieve temperature independence, the resistor ratios and other
parameters must be adjusted to result in ???????????? = 0
??????



?	Simple realization of the series bandgap form:
?	Assume ????????  = 0 and an ideal opamp
V	? VEB1 ?VEB2


? U ? ln?	I1	? ? U




? ln?	I2	?



T	? A	? I? ?	T	? A	? I? ?
?	E1	S ?	?	E 2	S ?


? U ? ln? I1 ? AE 2 ? ? U




? ln? R1 ? AE 2 ?



T	? A	? I ?	T	? R ? A	?
?	E1	2 ?	?	3	E1 ?
?	The opamp ensures that the voltage 
drops across ??3 and ??1 are equal,
i.e. ??1 ? ??3 = ??2 ? ??1


==> V	? V	? I ? R ? V	? V	? R1 ? V	? R1 ?U




? ln? R1 ? AE 2 ? ? V




?	R1 ?U




? ln? R1 ? AE 2 ?



REF	EB1	1	3	EB1


R2	R


EB1	R


T	? R ? A	?


CTAT	R


T	? R ? A	?


2	2	?	3	E1 ?


2	?	3


E1 ?


?	Differentiating w.r.t. to ?? and equating the result to zero yields



R1 ? ln? R1 ? AE 2 ? ? VG0 ?VCTAT ? ?

? 


? UT


R	? R ? A	?	U
2	?	3	E1 ?	T



?	For ??OS ? 0, ??REF becomes:


?	R ?	R


? R ? A	?	V	? ?


V	? V


? ?1? 1 ? ?V	?   1 ?U



? ln?   1	E 2 ?1?	OS    ? ?


REF	BE1


R2 ?	R2


? R3


? AE1 ?	I


2 ? R1 ? ?




? 

?	Recap: A temperature independent voltage can be obtained by adding 
the PTAT and CTAT voltages
??REF: temperature independent (bandgap) voltage
??EB: CTAT voltage
?? ? ???EB: PTAT voltage
?	New: The ratio of the PTAT and reference voltage ??REF can be used as 
a measure of temperature




?	Using a current mirror ratio of ??, an emitter area ratio of ?? = ????2/????1
and an amplifier gain of ?? results in


VPTAT


? K ? kT
q


? ln?p ? r ?


?	The CTAT voltage is obtained from ??EB2, the ADC computes the ratio 
of ??PTAT and ??REF to obtain ??




?	Why don\textbullet t we just measure ??PTAT because it is already proportional to 
absolute temperature?
We could certainly do that. However, we want a digital output and 
every ADC needs a reference voltage
The smartness of the sensor comes from the fact that we use a 
temperature independent quantity (the bandgap voltage) as reference 
for the sensor
Moreover, when choosing the (digital) ADC offset (??) and gain (??) 
correctly, the digital codeword is directly a digital temperature reading






Pertijs et al., JSSC, Dec. \textbullet 05


?	CMOS process	use substrate PNPs
?	Spread in ????	use one room temperature trim to obtain an absolute 
temperature measurement
?	Inaccuracy < \textbullet 0.1 \textbullet C (3??) from ?55 \textbullet C to 125 \textbullet C (military range)



?	Capacitive sensors are very suitable for displacement measurements


?	Parallel plate capacitance approximation


C ? ?


? ? ? A ? ? ? A , ? ? 1 for air



0	r	h	h	r



1	2	3






C ? ?


? w ?x0


? ?x ?



C ? ?


?	A	



C

? ?


?	A	



?C ? C ? C0


h
? ? ? w
h



? ?x



C0 ? ?


h ? ?z
? A h


top


Cbot



? ? ?


h ? ?z A

h ? ?z


Linear	Nonlinear	Reduced nonlinearity due to 
differential measurement
3



Charge amplifier


vout


? Csense
CFB


? vex








Voltage amplifier



vout


? Av ? C


Csense
?	C


vex


sense	ref




Transimpedance amplifier

vout ? s ? RFBCsense ? vex




?	The node ??sense has no DC connection to gnd (it is a \textbullet floating node\textbullet )
The DC potential is not well defined
\textbullet 	The node is very sensitive to leakage currents charging/discharging the node
?	A bias resistor provides a well-defined DC bias point
The value of ??BIAS  needs to be large because ??BIAS together with the 
capacitors forms a high pass whose corner frequency needs to be sufficiently 
low compared to the excitation frequency to avoid signal attenuation and 
discharge during the periods where the excitation voltages stay constant
\textbullet 	The large RC-time constant associated w/ large values of ??BIAS causes a 
slow discharge of the node against drifts
?	We can also use a periodic reset using a switch (reset requires 
introduction of reset phase, where ??p+ = ??p? = 0)




A. Burstein and W.J. Kaiser, \textbullet Mixed analog-digital highly sensitive sensor interface circuit for 
low-cost microsensors\textbullet , Intl. Conf. On solid-State Sensors and Actuators, June 1995
?	The circuit uses correlated double sampling to reduce offset, charge 
injection and 1/f noise
?	During ????1  the excitation voltage is applied to the bridge	??sense 
changes its value according to the change in capacitance of ??sense1,2 
the corresponding voltage is sampled on ??1
Note: ??sense also contains errors due to charge injection, amplifier offset 
and amplifier 1/f noise
?	During ????2 zero volts are applied as excitation voltage to the bridge
??sense now contains only changes due to errors (charge injection etc.) 
This voltage is stored on ??2 and subtracted from the voltage on ??1






?	When using a voltage amplifier to readout the capacitive difference between
????1 = ?????? + ???/2 and ????2 = ?????? ? ???/2, ??out becomes very sensitive to 
parasitic capacitances at the center node
v	?	?C	? A ? v



out


2C	? C	? C


v	in


s0	p1	p2
?	Parasitic capacitances associates with interconnects can be bootstrapped out
by using a voltage buffer driving the shield potential ??shield


v	?	?C


? A ? v


out	2C	? C	? C


v	in


s0	p1	p2

?	Intrinsic sensor parasitics can typically not be bootstrapped out (e.g. the 
capacitance associated with the anchor area of a released MEMS)



?	In the TIA (transimpedance 
amplifier) readout circuit the 
effect of ??p is attenuated by the 
gain of the opamp ??(??)


v	?	1	?


s ? Csense ? RFB




x	1? A ?s ?	?C	? C ?R	



ierror


1
? 
s ?

? s ?Cp ? vx


sense	p	FB
1? A ?s ?



?	Also in the charge amplifier, the 
virtual gnd node introduced by 
the opamp mitigates the effect 
of ??p


vx ?


?C CFB


?	1
1
?

?
C

?

C
p

?

A

?
s

?

C
F
B


ierror


? s ?Cp ? vx



?	Assuming an ideal opamp


?vin,op


? 0,


iin,OP


? 0 ==>


iin ?


j ? ? ?Cin ? vin ?? ?


iFB


? ? j ? ? ?CFB ? 
vout ?? ? ? iin





?	??in can be the sensing capacitance of a capacitive sensor
?	Then, for lateral displacement sensing ??in = ???? = ???? + ???, ??? ?? ???

?	For vertical displacement sensing ???? = ???? + ???, ??? ?? ???/(1 + ???)




?	For an ideal opamp:

?	In practice:
\textbullet 	The opamp is non ideal (finite DC gain, finite GBW and finite input 
impedance)


\textbullet 	The opamp can have a nonzero DC bias current 
to avoid opamp saturation a
discharge of ??FB is required 
(continuous or discontinuous)



?	Frequency response of the charge 
amplifier with finite gain and GBW opamp







?	In a typical charge amplifier, 
the excitation voltage ??in is 
attenuated, i.e. ??out < ??in 
because ??in ? ??FB
Features
?	Excitation voltages up to the 
opamp\textbullet s GBW are possible
?	The negative opamp input is a 
virtual gnd and can thus be 
used for adding currents in a 
differential sensor structure






?	The differential readout topology shown above is frequently used to 
read out MEMS structures such as accelerometers and gyroscopes



?	The output signal is proportional to the half-bridge excitation voltage \textbullet ??ex
any error in \textbullet ??ex will directly translate into an output error 
Accuracy is particularly important when sensing small values of ??? 
Mismatch (or drift) of the two amplitudes will introduce errors
?	For sinewave excitation the signals \textbullet ?????? 
can be generated using a center-tapped 
transformer or an active balun. This 
configuration is called a Blumlein bridge


?	For rectangular excitation signals and 
switched excitation schemes, the signals 
can be generated using the scheme below



?	Assuming ideal opamps (??op ? ?)


==> v x2


? v x1,


v x3 ? 0



?	The part of the circuit highlighted in 
red is a simple inverting amplifier 
with input voltage ??outn and output 
voltage ??outp
?	The part of the circuit highlighted in 
blue is a simple non-inverting 
amplifier with input voltage ????1 and 
with its reference potential pulled to
??outp
?	The part of the circuit highlighted in 
green is a voltage divider between
??in and ??outp



?	From the inverting amplifier subcircuit it 
directly follows


?	The noninverting amplifier subcircuit 
enforces the following relations



voutn


?	vx1


? vx1 ? ??voutn ? ==>


R	R
vx1 ? 0


?	The resistive divider finally enforces the 
following relation between ??in and ??outp


v	? v


vx1 ? 
0
x1	outp


in	x1 ?	==>
R	R





A. Burstein and W.J. Kaiser, \textbullet Mixed analog-digital highly sensitive sensor interface circuit for low-cost 
microsensors\textbullet , Intl. Conf. On solid-State Sensors and Actuators, June 1995
?	Idea: Introduce FB to regulate either the positive (shown) or the negative 
supply to provide a truly balanced excitation
?	Working principle: Use a pair of reference capacitors ????1, ????2 (matched to ????1 
and ????2) to measure the mismatch between the positive and negative 
excitation voltage
?	The FB loop enforces ??ref = 0 by adjusting the magnitude of the positive 
excitation voltage




?	Capacitance to displacement relation for

?	Linearization for small displacements

Output is approximately 
proportional to displacement!






















?	For open loop readouts typically a lock-in 
detection scheme is used to maximize SNR
?	For large excitation signals the sensor 
non-linearity becomes an important 
issue
?	This is true for all open loop sensor 
readout systems




Integrated Interface Circuits \textbullet Chapter 2: Sensor types





?	Using an inductor, we can embed any capacitive sensor into a 
resonant circuit w/ resonant frequency ????res  (Note: The self resonant 
frequency of the sensor has to be significantly larger than ????res)




Q ?
Rcoil ? Rloss
?	? 1 ? ?	?

res	2	LC

?LC ?


M. Lemkin and B. Boser, \textbullet A three-axis micromachined 
accelerometer with a CMOS position-sense interface and 
digital offset-trim electronics, JSSC, pp- 456-468, April 
1999










?	Differential readout topologies greatly suppress common mode 
interferences (power supply noise, charge injection, substrate noise)
?	Using the differential charge integrator on the top right with the drive 
signal applied to the proof mass, only one excitation source is required
?	The output signal of this structure is given by
?	?


? ?C	?


C	?	?C


?	?C	C	?


?V	? ??V


? ?	s ? ?1?	s
? ?	i	p 
?	s	?


out	ex	? C


?	C ? C ? C ?


C	C ? C ? C ?


?	i	?


\textbullet \textbullet s


\_\textbullet i


\textbullet \textbullet ,p ?


\textbullet \textbullet i


\textbullet \textbullet \_\textbullet s


\textbullet \textbullet i


\textbullet ,p ?


?	gain error


offset error	?


?	?














M. Lemkin and B. Boser, \textbullet A three-axis micromachined 
accelerometer with a CMOS position-sense interface and 
digital offset-trim electronics, JSSC, pp- 456-468, April 
1999


?	W/o the input common mode feedback (ICMFB), the opamp\textbullet s input 
common mode voltage would make steps ???iCM in response to the 
steps in ????	The opamp needs a large input CMR and CMRR
?	The CM feedback amplifier senses the input common mode voltage 
of the main opamp, compares it against a reference and applies the 
required voltage at the foot point of the bridge formed by ????1, ????2, ????????1 
and ????????2 required to bring the input common mode voltage back to its 
nominal value after every jump


Parallel plate capacitor	Comb finger capacitor






C ? ?



?	A	



?C ?x 
? ? ? 




? ? A


N
:



f
i
n
g
e
r
s
h
:

f
i
n
g
e
r

h
e
i
g
h
t
C ? x 
? ? N 
? ? ? h 
? ?L ? 
x ?



?C ?x 
?



? ?N ? 
? ? h


x0 ? x


?x	?x0


? x ?2


g	?x	g






Capacitive structures can be used for both sensing and actuation. We can 
use this to form feedback systems around a capacitive sensors!





















?	In a capacitive sensor with comb structure, we typically exploit the 
possibility of realizing differential capacitances
?	We either need dedicated electrodes to apply the force or we can use the 
same electrodes for sensing and actuation by time multiplexing the 
electrodes (in this case ??s1 = ??a1, ??s2 = ??a2)
?	??FB is the voltage applied to generate the electrostatic feedback. For ??FB = 
0 the net force is zero and the plate moves back to the center position
?	In contrast to the single-ended case, here the voltage-force relation is linear



?	Modeling the mechanical resonator
















C0 : static capacitance


Index \textbullet m\textbullet identifies motional quantities











?	Assuming a parallel plate structure for the sensor, we obtain:



Fmech : mechanical force (e.g. gravity)
Vn : noise in the voltage across the capacitor



Vn ?t ? ?Vn ?t ? ? ?



? 2 ? kBT ? R ? ? ?? ?


Fn : mechanical force noise


Fn ?t ? ? Fn ?t ? ? ?


? 2 ? kBT ?? ? ? ?? ?



?	Nonlinear dynamics (nonlinear in the voltage only)
?	Coupling between electrical and mechanical noise
?	In a nonlinear system, the superposition principle does not apply
?	To obtain the intrinsic sensor SNR, we only have to consider the force 
noise ???? and compute its effect on the sensor output quantity!



?	The transfer function from the mechanical noise to the displacement ?? is





?	The displacement noise PSD due to the force noise ????
const. then becomes:


= 2?????????? =








?	Displacement noise due to 
mechanical force noise at ????:

?	Displacement noise due to 
mechanical force noise at ?? = 0:


?	Low frequency noise decreases by 
increasing the quality factor ???? (i.e. by 
decreasing friction)	here 
vacuum encapsulation helps
?	BUT: Noise at the resonance 
frequency increases with ????!



?	Signal (i.e. displacement in response to an applied force)


?	Signal is shaped in the same way as the noise








?	Synchronous demodulation delivers magnitude, phase, real part and 
imaginary part of the frequency response in one frequency sweep!





?	Why feedback?



Y ? A ? X









Y ?	A	? X

1? ? A
? 1 ? X
?

For large loop gains ?? ? ??

?	In a feedback system with sufficiently large loop gain, sensitivity and 
linearity are NOT determined by the sensor ?? but rather by the 
feedback network. This is typically easier to achieve!





?	Feedback around the sensor can be used to enhance the bandwidth
?	Feedback (FB) does not improve the SNR per se
?	However, it allows for the use of very high-Q structures
thereby FB indirectly helps to lower the Brownian motion noise 
(the random motion of the structure)





?	Typical open loop response of the sensor is a resonant 2nd order system
?	What type of controller should you use for the FB?
\textbullet 	P-controller alone not a good idea because of insufficient phase margin (PM)
\textbullet 	PD-controller (lead compensation) introduces additional PM



KP ? 50,


K	? 5 ?10?3


\textbullet 	Could add an I-controller part (full PID-controller) to null steady state errors




?	The feedback signal is used to keep the comb structure at nearly zero 
displacement	minimum effect of nonlinearity
?	The same electrode is used for sensing and applying the feedback 
signal. How is this possible?	Frequency domain separation of the 
feedback and the sense signal (synchronous demodulation)



?	For a single-ended capacitive structure, the voltage-force relationship 
is non-linear (cf. slide 98)


?	However, we would like to have a linear voltage-force relationship to 
obtain an overall linear feedback loop
we could either use a nonlinear controller which cancels the 
nonlinearity (i.e. a square root controller)

Or we can use a force feedback with two values only
\textbullet 	The connection between two points is always
a straight line (only gain error and offset possible)
\textbullet 	Can apply a two point force feedback by exploiting 
the sigma-delta modulation technique
\textbullet 	Pulse-density modulation rather than amplitude 
modulation









C. Lu, M. Lemkin, and B.Boser, \textbullet A monolithic surface micromachined accelerometer with digital 
output\textbullet , JSSC, pp. 1367-1373, Dec. 1995



?	The compensation block is required for system stability (essentially a 
digital lead compensation)
?	Also, the compensation block can be used to optimize the noise 
shaping performance of the closed system




Analog feedback
?	Continuous feedback 
minimizes ??in ? ??FB
?	The output is an analog 
signal
Digital feedback
?	?? minimizes mean of
??in ? ??FB
?	Mean of ??FB and ??out
track ??in
?	??in can be recovered by 
low pass filtering ??out
?	The digital feedback 
system is clocked



?	Measuring the capacitance value using the charge redistribution method
\textbullet 	Step1: S3 closed, S1 to ??ref, S2 to gnd, DAC output is set to 0
????  forced to gnd by the amplifier in feedback, ????  is charged to ??????  = ??ref  ? ????
\textbullet 	Step2: S3 opened, S1 to gnd, S2 to ??ref, ??DAC = 0
S3 open freezes charge on node ????, total charge on node ????:


Q ? v ?C ? ?v ?V	? ?C ? v ?C	? ?Q	? ?V	?C	==> v	? V	?


CS ? CR



x	x	R	x	ref	S	x	C	R	ref	R	x	ref


C ? C ? C


\textbullet 	Step3: Adjust ??DAC until ???? = 0


S	R	C


Qx  ? vx  ?CR  ? ?vx  ?Vref ? ?CS  ? ?vx


?	vDAC ? ?CC


? ?QR


? ?Vref


?CR











Integrated Interface Circuits \textbullet Chapter 2: Sensor types



?	Embed a differential capacitive sensor into a relaxation oscillator













v1 ?t ? ? sgn ??vc ?t ???


vc ?t ? ? v1


?t ? ? ??v


2 ?t ? ? v1


?t ??? R  ? R


7	8


v ?t ? ? ?	1	?	v ?? ?d? ?


1	?	v




?? ?d?


2	RC	? 1


RC	?


out


3	1	1	1
R	dv2 ?t ?
v4 ?t ? ? ?	2 ? v1 ?t ? ? R2C2 ?
R4	dt
R
vout ?t ? ? ?	6 ? v4 ?t ?
R5



?	Embed a differential capacitive sensor into a relaxation oscillator



































Integrated Interface Circuits \textbullet 
Differential Measurements


Jens Anders
Institute for Theory of Electrical Engineering 
Department of Electrical Engineering and Computer 
Science
University of Stuttgart



?	Differential signaling has many advantages compared to single-ended 
signaling
\textbullet 	Inherent rejection to common-mode interference and noise
\textbullet 	Wider signal swing for a given supply voltage
\textbullet 	Minimum effect of even order distortion including DC offset
\textbullet 	Can lead to improved sensor linearity


?	BUT most practical amplifiers also respond to some degree to 
common-mode signals at their inputs
\textbullet 	The relevant figure is the common-mode rejection ratio (CMRR)
\textbullet 	Good amplifiers for high-end sensing applications have CMRRs better than 
120 dB



?	A simple diff amp consists of a differential pair with resistive load
?	Its output voltage can be written as
?	Assuming ideal components (i.e. no mismatch
and an ideal current source) this structure provides:
\textbullet 	No offset
\textbullet 	An infinite CMRR and PSRR
?	Differential pairs can be easily built using both MOS 
and bipolar transistors
?	Definition of CMRR


?	Definition of PSRR



?	In practical differential amplifiers component mismatch (??1 ? ??2, ??1 ?
??2) and imperfect current sources (??out? ?) lead to
\textbullet 	Offset
\textbullet 	Finite CMRR and PSRR
?	Mismatch is mainly due to process variation lithographic errors





?	Intuitive interpretation:
the input referred offset voltage is the voltage, which needs to be 
applied to obtain zero volts differential output voltage











v	? ? R


? ?RL ? ? ? I




? ?ID ? ? ? R




? ?RL ? ? ? I	? ?ID ?

? 


out	?	Lav	2	? ? Dav	2 ?	?	Lav	2	? ? Dav	2 ?
?	? ?	?	?	? ?	?





?	Next, we need to approximate ????? = ????1 ? ????2



?I	?


?1 ? ?V	?V



?	nV


?2  ? ?2


? ?V	?V


?	nV  ?2


D	2n


G1	T01


S	2n


G 2	T02	S



?	Same idea as before: Introduce


??    ?


? ? , ?


?1 ? ?2 
, ?V	V



?V	, 
V


VT01 
? VT02 , 
?V	V



?V	, 
V


VG1 ? 
VG2



1	2	av	2


T0	T01	T02
T0av	2


G	G1	G2
Gav	2


?av


? ??	? ?


?V ?	?


?V	?	?2


?I	?	2 ?	V


?	 	G ? V


?	T0


?	nV


D	2n


? ??


Gav


2	??


?? T0av


2	??	S ?


?	?


?av


? ??	? ?


?V ?	?


?V	?	?2


?	2 ?	V


?	 	G ? V


?	T0


?	nV


2n	? ??


Gav


2	??


?? T0av


2	??	S ?


?	?


?ID


? ?ID ?? ?0,?V	?0,?V ?0 ?
? 0


?? ?0,?VT0 ?0,?VG ?0


? ?? ?


?? ?0,?VT0 ?0,?VG ?0


? ?VT0 ?


?? ?0,?VT0 ?0,?VG ?0


? ?VG



?	Evaluating the expression from the previous slide, we find:



?ID ?



1 ? ??



?	Gmav


? ?V


?	Gmav


? ?V , G


? ?IDav



I	?	I


T0	I


G	mav	?V


Dav	av	Dav	Dav	Gav





!
vout ? 0 ==>
0 ? R	? I



? ? ??



?	Gmav




? ?V



?	Gmav




? ?V



? ? I




? ?R


Lav	Dav	? ?	I


T0	I


G ?	Dav	L


?	av	Dav	Dav	?

?	Now, we need to solve this equation for ???G = ??OS, resulting in




?	Finding from previous slide

?	??mav/??Dav is an important parameter which largely varies vs. the 
operating point of a MOSFET
?	Moreover, the above derivation also holds true for BJTs with the


difference that ???????


? 0 for BJTs (????


= ????
??


is equal for both devices)



?	The achievable offset voltages can thus be summarized as follows:



10 mV	?	2
??ov


1
? 100 mV


MOS (SI)


1 mV

120 ?V


1
?	? 1/26 mV
??
no change


BJT

With offset 
trimming 
(Laser, 
fusible links)



?	In a real-world differential pair mismatch and finite output impedance of 
the current source also lead to finite CMRR
?	As before, decompose all quantities according to



?G	G


?	G	, G


Gm1 ? Gm2



m	m1	m2	mav	2


?R    R


?	R , R


RL1 ? RL2



L	L1	L2	Lav	2
?	Then, performing a simple nodal 
analysis yields



?	Using the result from the previous slide, the CMRR of a diff pair 
becomes



CMRR 20 ? log?
A


ADM	?
?



ADM


?

? ?Gmav ? 
RLav


CM-to-DM ?






ACM-to-DM


? ? ?Gmav  ? ?RL
2G


?	RLav ? ?Gm ?
? r


mav	out


? ? 



?	Relating ???m/??mav to physical parameters, we obtain:


G	?	, I


? ?av


? ?V ?V


?	nV  ?2



?G	?


?Gm


? ??


?	?Gm ?




?ID	?V




mav


Dav	2n


G	T0	S


m	??


?I	?V	T0


D	T0





?	Note:


Question: Is there any link between CMRR and offset?



?	Expression for ??OS assuming random fluctuations in ?????, ? ?? and ??????? 
the signs of the individual terms do not matter (worst case all are 
added)


?	Expression for (linear) CMRR assuming random fluctuations in ?????, ? ??
and ??????? (again worst case all are added)

?	These two expressions look quite similar, ignoring the little difference


caused by the factor of 1
2


and multiplying the two 
expressions, we find:



?	From previous slide
?	Note: ??mav ? ??ov = ??BIAS



?	Recall: ??out


=	????
(??BIAS/2)


where ????


is the modulation 
voltage 
(modeling


channel length modulation)
?	Note: Our derivation actually also holds true for BJTs and MOSFETs in 
weak inversion




?	The power supply rejection ratio(s) is/are defined as:




Slide 16	PSRRVDD calculation
?	First let us calculate the PSRR for the positive supply, PSRRVDD
?	As long as the finite output conductance of
??1 and ??2 are neglected, mismatch in ??L1,2 
and ??1,2 result in no differential output 
voltage, i.e. infinite PSRRVDD
?	Assuming:



RL1 ? RLav


?	?RL,


RL2


? RLav


?	?RL


Gm1 ? Gmav


?	?Gm,


Gm2


? Gmav


?	?Gm



and finite output conductances of ??1,2
of ??outM1,2, the resulting PSRR becomes

resulting in a PSRRVDD of




Slide 17	P????RRV????????  calculation
?	Assuming:



RL1 ? RLav


?	?RL,


RL2


? RLav


?	?RL


Gm1 ? Gmav


?	?Gm,


Gm2


? Gmav


?	?Gm



and finite output conductances of the current 
source ??out, the resulting PSRR becomes




?	Recall from the previous slide:









Integrated Interface Circuits \textbullet 
Modulation techniques to improve 
low frequency performance

Jens Anders 
Institute of Smart Sensors
Department of Electrical Engineering and Computer 
Science
University of Stuttgart



?	Synonyms: coherent detection, synchronous demodulation, lock-in 
amplification, chopping
?	These are all modulation techniques that are used to improve the low 
frequency performance of measurement systems
?	When square-wave modulation is employed the technique is referred 
to as chopping
?	Chopping leads to improved low-frequency specifications e.g. 
reduced offset and 1/???? noise, better CMRR and PSRR



?	Characterized by
\textbullet 	Offset, gain error
\textbullet 	Drift, 1/???? noise
\textbullet 	PSRR, CMRR
?	Offset, gain error and 
1/???? noise are caused  
by component mismatch 
and non-idealities
they are a part of life
?	But we can reduce their effects by
\textbullet 	Static techniques like calibration and trimming
\textbullet 	Dynamic techniques such as chopping, auto-zeroing 
and dynamic element matching



?	Involves measuring a static error of a system
(e.g. offset or gain error) and then adjusting the value of a component 
in order to reduce the error to zero
+ Low complexity
+ No bandwidth limitation
- Requires measurement equipment
?	Also requires a memory element 
to store the trimmed value
e.g. a potentiometer or a PROM
C






Vin



Vout



?	Techniques which continuously attempt to cancel the effect of
system non-idealities to zero.
+ (Usually) do not require measurement equipment
+  Also compensate for drift and 1/???? noise and improve 
CMRR and PSRR
-	Requires more complex circuitry
-	Reduce bandwidth


?	Two main Dynamic Offset Cancellation (DOC) techniques are:
Chopping and Auto-zeroing



?	Time domain: Auto-zeroing
periodically measure the offset (the noise) and subtract it from the 
input signal
?	Frequency domain: Chopping
modulating the input signal above the 1/???? noise



?	Signal is modulated, amplified and then demodulated
?	DC offset is modulated once and the resulting AC signal can be 
removed by a low-pass filter
?	The modulators are usually implemented as polarity reversing 
switches, known as choppers
?	The technique is known as chopping








?	Easily generated modulating signal
?	Modulator is a simple polarity-reversing switch
?	Switches are easily realized in CMOS








?	Output chopper converts offset into a square-wave
?	To avoid residual offset, the duty-cycle of the square-wave 
should be exactly 50%
?	Non-ideal LPF	residual ripple






?	Complete suppression of 1/???? noise if ????ch  > 1/???? corner frequency


but harmonics	slightly (??2
8


? 1.23) higher noise power


because noise from higher harmonics folds into baseband
?	Up-modulated offset must be filtered out
loss of signal bandwidth and residual chopper \textbullet ripple\textbullet 













xsquare ?t ? ?


4 ?
?	k ?1


sin?2? ?2k
2k


? 1?? fch
? 1


? t ?



?	Turning off/on the switch injects charge in ??L
?	How much of the charge is added to ??L?
\textbullet 	Usual assumption:
The charge splits half-half
\textbullet 	In reality the amount of charge is a complex function of
the impedance at each terminal and the transition time of the clock
?	???? is linear (proportional) w.r.t the transistor area ???? ? ?? and nonlinear
w.r.t. ??in


Q ? W


? L ?C	? ?V


?Vin


?VT0 ?	??


?	?
?	??out can be decomposed into a term linear w.r.t ??in and a 2nd term 
nonlinear w.r.t ??in
Distortion	?	?


V	? V


?1? W


? L ?Cox


? ? W


? L ?Cox


? ?V ?V


?	??


out


in ?




2C	?	2C


? ?	T0	?


\textbullet \_L\textbullet ,
?	gain error ?


\textbullet \textbullet L \textbullet \textbullet \textbullet \_\textbullet \textbullet \textbullet \textbullet \textbullet ,
offset , dostortion



?	Capacitive coupling due to ??OV
capacitive divider between
??OV and ??L

?	Clock transition couples
to ??out and creates an error

?	???out is independent of ???in



?	???out is linear w.r.t capacitive divider ratio
constant offset




?	Clock asymmetry (non-50% duty-cycle)
?	Spikes at the amplifier\textbullet s output due to clock feedthrough and charge 
injection
?	Demodulation of these spikes back to DC by the output chopper 
residual offset (in the range of millivolts)
?	Residual offset is proportional to the chopping frequency







?	Limited amplifier bandwidth results in an output signal which is not a 
perfect square-wave
less power at the harmonic	less gain
?	Chopping reduces DC gain









?	Chopping is a powerful technique that can be used to reduce offset 
and 1/???? noise in amplifiers and systems


?	Main drawback: the need for a LPF to remove the up-modulated offset 
bandwidth limitation

?	Main non-idealities are caused by:
bandwidth, clock asymmetry and chopper spikes


?	Offsets as low as a few nV can be achieved



?	Three-stage amplifier, first two stages are chopped
?	1 mHz 1/???? corner frequency, 5 \textbullet V offset voltage, 
CMRR , PSRR > 120 dB



R. Wu, K.A.A. Makinwa and J.H. Huijsing, "A Chopper Current-Feedback Instrumentation Amplifier With a 1 mHz 
Noise Corner and an AC-Coupled Ripple Reduction Loop," Solid-State Circuits, IEEE Journal of , vol. 44, no. 12, 
pp. 3232-3243, Dec. 2009


Analog Storage	Digital Storage



Vin



Vo
ut



Vin


?2 
S
?1


A
1

N
Vn	CLK



V
o
u
t
A2



SAR = Successive
Approximation 
Register



DAC
?	Sampling the unwanted signal (noise and offset) during ?1
?	Storing this value (analog or digital)
?	Subtracting the stored value from the input signal during ?2
?	Amplifier is disconnected from signal path during ?1
?	An auxiliary port N can be used to apply the stored value 
offset and low-frequency noise can be greatly reduced


Vref



?	S1 and S2 are closed
??out = ??OS
?	The amplifier\textbullet s offset 
is stored on ??az

?	S3 is closed
output signal 
is available
?	For an amplifier with finite 
DC gain ??, the residual 
offset is given by ??OS/(?? + 1)



?	Clock feedthrough and charge injection 
errors in stored offset





?	Stored offset on ??az will slowly leak away

?	In practice, ??az  is made as large as possible



?	Use minimum size switches
(subject to noise , speed requirements)


?	Use differential topologies 
for 1st order cancellation


?	Use complementary switches


?	Use dummy switches


?	Use bottom plate sampling





?	The error in the differential output signal can be written as:


?Q2


?	?Q1  ? W ? L ?Cox  
???Vinp


?Vinn ? ? ?VT02  ?VT01 
???


? W ? L ?Cox  ??Vinp


?Vinn ? ? ? ? 2?F


? VT02 ?


2?F


? VT01 ??


?	?
perfect cancelation is only achieved for zero differential 
input signal (??inp = ??inn)

?	BUT: The constant offset as well as the nonlinearity due to the 
square root term are reduced compared to a single-ended 
switch




?	Use an NMOS and a PMOS switch in parallel
we can achieve a first order cancelation of the opposite 
charge packets. To this end we need:
Consequently, the drain source cancelation can only occur 
for one input level ??in
?	Also for clock feedthrough, there is not a perfect cancelation due 
to the different overlap capacitances of NMOS and PMOS





?	Assuming the charge injection due to ??1 is:
?	Then, since the charge injection of ??2is

for ??2  = ??1  and ????2  = 0.5 ? ????1  the two charge packages cancel
?	Unfortunately, the assumption of equal charge splitting of ??1 is in 
general not valid
?	Fortunately, the above choice of ??2 also mitigates clock feedthrough



















?	Switch ??2 opens slightly earlier than ??1
the charge injected by the opening of ??2 is constant and can be
eliminated by using differential sampling
?	Since the bottom plate of ??az is floating when ??1 is opened, no charge 
is injected on ??az
?	This scheme works very well but requires an additional clock signal


Slide 29	Effect of Autozeroing on Noise
VC
R




VN



When switch S is open:


VAZ


VAZ


? VN


?VC


for: RC


\textbullet TAZ



?	Each time the switch ?? is closed, the output voltage ??AZ is reset 
to zero and the noise source voltage ??N appears across resistor
?? and capacitor ??
?	Assuming ???? ? ??AZ at the end of the sampling phase (when 
switch ?? opens) the noise voltage is sampled onto capacitor ??
?	The output voltage ??AZ becomes equal to the difference between 
the instantaneous voltage ???? and the voltage ???? stored on ??
DC component of ???? is eliminated but HF is passed



?	If the source voltage ????(??) corresponds to a stationary random noise with PSD
??????(????), the PSD of the autozero voltage ????????(??) across the switch can be 
decomposed into two components:
\textbullet 	One caused by the baseband noise (which is reduced by the autozeroing process)
\textbullet 	And the other one created by the foldover components introduced by aliasing:



S	?f ? ? H


?f ? 2 ? S


?f ? ? S


?f ?


VAZ


0\textbullet \textbullet \_\textbullet VN\textbullet ,
baseband


\textbullet fo\_ld \textbullet ,
foldover






where:


Sfold ?f ? ?


??

n???
n?0


2
Hn ?f ? ? S	?
?


n ?
?	?
Ts ?



?	The baseband transfer function is given by:


2	? ?


sin?2? ? f ?T  ??2

?


cos?2? ? f ?T  ??2 ?


H ?f ?	? d 2 ? ? ?1?


h ?	? ?1?


h ? ? ? ? ? f ?T	for
? ? f 
?T \textbullet 1



0	?	2? ? f


?Th	?	?


2? ? f


?Th	? ?



where ?? = ???/???? is the duty cycle (???? = ??? + ????????)






S	?f ? ? H


?f ? 2 ? S


?f ? ? S


?f ?


VAZ


0\textbullet \textbullet \_\textbullet VN\textbullet ,
baseband


\textbullet fo\_ld \textbullet ,
foldover







Sfold ?f ? ?


??

n???
n?0


2
Hn ?f ? ? S	?
?


n ?
?	?
Ts ?










?	The baseband transfer function imposes a zero at ???? = 0, which cancels 
out any DC component present in ????(??)
?	The transfer functions for ?? ? 0 are given by:
?	sin?? ? f ?T  ??2


H ?f ?	? ?d ?


h ?	for


n ? 0	and	T	\textbullet T


n	?	? ? f


?Th	?


AZ	h






S	f	?
AZ


H ?f ? 2 ? S


?f ? ? Sfold


?f ?






with:


SV ?


?f ? ?


S0

1? ?f fc ?








?	In order to use the previous analysis, we simply have to replace the voltage 
source ????  by the noise at the output of the AZ amplifier



?	The baseband signal is autozeroed by ??0	2 while the PSD of the
autozeroed signal is dominated by the foldover component



S	?f ? ? sinc2 ?? ? f ?T ? ?	S


? f ?


n ? ? sinc2 ?? ? f




?T ? ? ?? ? f


?T ? 1? ? S


fold


s	?
n
?
?
?
n
?
0


VN?  ?
?


?
Ts ?


c

s

c

0



Reference: C. Enz and G.C. Temes, Proc. IEEE, Nov. 1996







S	f	?
AZ


H ?f ? 2 ? S


?f ? ? Sfold


?f ?






with:


SV ?


?f ? ?	?


1


1? ?f fc ?











?	The baseband 1/f is eliminated by the high-pass filtering action of


??0	2


?	An additional foldover component is present in the autozeroed output




S	?f ? ? sinc2 ?? ? f ?T ? ?	S


? f ?


n ? ? sinc2 ?? ? f




?T ?


2 f	T	?


? 2 ? f




?T ?? ? S


fold


s	?  VN?  ?	T  ?


c	s ?	? k ?


s ? ?1? 
ln? 3


c	s 
??	0


n???	?	s ?
n?0


?	?
??




Reference: C.C. Enz and G.C. Temes, Proc. IEEE, Nov. 1996



?	Auto-Zeroing is a powerful offset and 1/f noise reduction technique for 
amplifiers and systems
?	Unlike chopping, it does not suffer from ripples, but its noise 
performance is worse due to aliasing
?	Main non-idealities are caused by switching spikes, leakage currents 
and (sometimes) by finite gain
?	Offsets of a few microvolts can be reached






?	CDS is a special case of AZ frequently used in SC circuits
?	SC circuits are sampled data systems	clock available
\textbullet 	Phase 1: calibration phase: ??1 = ??(??1) = ?? ?
\textbullet 	Phase 2: operation phase: ??2 = ??(??2) = ?? ?
(??2 ? ??1) = ?? ? ??in

?	To maximize suppression of 1/f noise, the interval ??1 ? ??2 should be as 
short as possible





?	For small signals, offset dominates (can use AZ, CHS, \textbullet )
?	For large signals, gain error dominates








?	Both gain error and offset are static errors, which can be removed by 
calibration and/or trimming
?	It can also be removed by dynamic element matching (DEM)



?	If ??	? ?? ? 1	??CL ? 1/??



?	For moderate 1/??, DC-gain of op-amp is large enough

?	??: Resistor/Capacitor ratio
\textbullet 	Resistors: 0.01%, 5 ppm/\textbullet C
\textbullet 	Capacitors: 1%, 500 ppm/\textbullet C



?	So feedback allows us to replace the inaccurate gain of open loop 
amplifiers by well defined ratios of passive components	Can we 
do better?




?	Idea: In a sensor system we would like to know ??in but we do not know 
(precisely) ?? and ??OS	for 3 unknowns, we need 3 equations
?	3 Phases are required for a measurement:
\textbullet 	Phase 1: ??1 = ?? ?
\textbullet 	Phase 2: ??2 = ?? ?
\textbullet 	Phase 1: ??3 = ?? ? ??OS
??, ??OS and ??in can be found
Accuracy is limited by ADC resolution and noise



?	Gain errors can be further reduced by using Dynamic Element 
Matching (DEM)
?	DEM involves swapping the position of nominally identical elements 
within a circuit
?	The average error is significantly reduced
?	Gain of 2 ? 2 identical resistors
i.e. ??1 = ??2
?	DEM can be applied by using 
switches to swap the position
of mismatched resistors in the circuit
?	Accuracy is limited by mismatch 
of switch resistance





?	Accurate gain 2 amplifier with DEM
?	Average value of ??out ? 2??in
?	??out contains AC components which must be removed by a LPF 
(like in chopping)
?	Chopping and DEM can be easily combined






?	Mismatch is shifted to the harmonics of ????DEM
?	To avoid unwanted intermodulation, keep ????in  < ????DEM/2



?	Pro
\textbullet 	Gain (ratio) error can be reduced to ppm levels
?	Cons
\textbullet 	Switches are required to swap components
extra circuit complexity, switching transients
\textbullet 	Result must be averaged	BW reduction
\textbullet 	Input signal must be band-limited (????in  < ????DEM/2) 
to prevent inter-modulation products



?	Precise gain can be achieved by feedback
\textbullet 	discrete resistors: 0.01%
\textbullet 	on-chip: 0.1% due to mismatch
?	Better performance can be achieved by using DEM, 
but like chopping, this is at the expense of BW
?	If the signal is digitized, the 3-signal method is also effective, but 
accuracy is limited by ADC resolution








Chapter 5

Coherent Detection

Jens Anders 
Institute of Smart Sensors
Department of Electrical Engineering and Computer 
Science
University of Stuttgart



?	Review:
\textbullet 	Synonyms: coherent detection, synchronous demodulation, lock-in 
amplification, chopping
\textbullet 	These are all modulation techniques that are used to improve the low 
frequency performance of measurement systems
\textbullet 	When square-wave modulation is employed, the technique is referred to as 
chopping
?	Today we talk about coherent detection/lock-in amplification
?	This technique is very similar to chopping except that one uses a
sine wave instead of a rectangular waveform for modulation



?	When to use it?
\textbullet 	When measuring low-bandwidth or quasi-static signals in the presence 
of high noise or disturbance levels
\textbullet 	If a high dynamic range is required
?	Therefore it is often used to readout sensors such as:
\textbullet 	(MEMS) Accelerometers (fF of capacitance change)
\textbullet 	Optical (infrared) detectors (detecting fA\textbullet s of current change)
\textbullet 	Magnetic sensors (detecting mV\textbullet s of voltage change)
\textbullet 	Strain gauges (detecting mV\textbullet s of voltage change)
?	A coherent detector behaves like a band-pass filter



?	Setup:












?	??oM  = ??Amp ? ??mult?


???????  ? sin


? ?????  sin





?	??oM


= ??Amp


? ??mult


? ??????????????  ?
2


cos


? cos


?	For ???? = ?????? (?????? is derived from ???? ? coherent signals):



??oM


= ??Amp


? ??mult


? ??????????????  ?
2


1 ? cos









?	??oM


= ??Amp


? ??mult


? ??????????????  ?
2


1 ? cos


(????


= ??


????)


?	LPF to remove the sum frequency:


??oF


= ??LPF


? ??Amp


? ??mult


? ??????? ?????
2


?	A coherent detector directly extracts the amplitude of the input signal ??????
if ??i and ??ref are in phase (Synchronous detection)
?	What if they are not in-phase?

















?	??oM = ??Amp ? ??mult ?


???????  ? sin  ?????? ?? + ??	? ?????? sin  
??????


= ??Amp


??mult


??????? ???


2


cos


?  cos 
[????


+??????


]?? + ??



?	??oF




= ??LPF




? ??Amp




? ??mult



? ????????????
2


=0	LPF
? cos


?	Phase sensitive coherent detector



?	What would be the SNR of
?	an oscilloscope (wideband, LPF)
?	a coherent detector (narrowband, BPF)


?	Some numbers:
?	What would be the 
detection limit of the 
coherent detector if 
the white noise 
spectral density
is 10nV/?Hz and
fLPF = 1Hz?







????????


????V
?



= ???????? ????V














ui   ? u\textbullet i


? cos ??r t


? ? ?






?	Use a quadrature LO to obtain both the phase and the amplitude 
information










?	Linear operation of an analog multiplier circuit, working over a large
range of input signals is difficult to realize.
?	A much simpler realization 
is a switching detector 
aka a chopper
?	Chopper realization
in CMOS:




\textbullet 	Easy: just 4 Transistors
\textbullet 	Accurate: no offset introduced
\textbullet 	But: switching spikes can cause 
problems (residual offset)















?	An ideal coherent detector 
is only sensitive to 
frequency components
in a narrow band around 
a reference signal ??ref
?	The LPF determines 
the signal bandwidth

























A practical example \textbullet 
ESR DETECTION






Analyzing the structure and dynamics of \textbullet biomolecules\textbullet 
Source: SREP (http://sites.univ-provence.fr/srep/index.html)


Food industry: Quality control









Biomedical Imaging \textbullet Sensor Fusion


Source: S. Liu et al., Applications of in vivo EPR in brain research: Monitoring tissue 
oxygenation, blood flow, and oxidative stress


In-vivo ESR Dosimetry


Source: EPR Center Dartmouth College



?	What is \textbullet spin\textbullet ?
?	A property of some particles like mass and charge


?	What do we need to know to detect them?



?	A spin


???


always comes 
along with an 
angular 
momentum


???


and a magnetic


dipole moment


???





I\textbullet 
?: reduced Planck constant
????: gyromagnetic ratio






Parameters: 
Free electron


????
???????


? ?????????.024 GHz/T


??????  = ???????? ????????
??????  = ???????? ????T
? ??RF = ???? for
??????  = ????. ???? T



Circulator operation:
Non-reciprocal 3-port w/ S-parameter matrix:


? 0	0	1?
S ? ? 1	0	0 ?
? 0	1	0 ?


Incident power at one port is 
transferred to the next port











?	Sample absorption is a (resonant) function of the static ??????-field
?	Resonator containing the sample is critically coupled at Larmor frequency 
(critically coupled resonator ? no reflected power)
?	Static ??????-field is swept through the resonance field 
Resonant sample absorption changes cavity Q factor
Reflected power at circulator output varies as function of ??????
CW -ESR spectrum is the reflected power vs. ??????



ESR \textbullet DC\textbullet spectrum
Parameters: 
Free electron


????
???????


? ?????????.024 GHz/T


??????  = ???????? ????????
??????  = ???????? ????T
? ??RF = ???? for
??????  = ????. ???? T



?	ESR signal inherits the absorption shape (associated w/ sample loss)
?	Resonance condition: ??RF  = ??L  = ???????????



?	When sweeping ??



????


, the peak occurs at ??



????


= ???RF
????


ESR signal generation in the 
presence of field modulation


Resulting ESR signal 
for small 
modulation 
fields

























?	The \textbullet DC\textbullet ESR spectrum is relatively noisy due to low frequency
disturbances (e.g. 1/f noise and drift)
?	The signal is modulated to an IF by modulating the ??????-field
?	In this way, a lock-in amplifier can be used to extract the signal








Chapter 6
Integrated Interface Circuits \textbullet 
DACs

Jens Anders 
Institute of Smart Sensors
Department of Electrical Engineering and Computer 
Science
University of Stuttgart



?	Recap: digitization
\textbullet 	Time discretization (sampling)
\textbullet 	Value discretization (quantization)
?	Digital-to-analog conversion
?	Analog-to-digital conversion


























Integrated Interface Circuits \textbullet Chapter 6: DACs



?	Sampling is the conversion of a time-continuous signal into a value 
equivalent, time-discrete signal
?	Prerequisite: Nyquist sampling theorem:
To be able to perfectly reconstruct a band limited signal  ????max  = ?????? 
from an equidistantly sampled version (sampling frequency ??????  = 1/????) 
of it, the sampling frequency must satisfy ??????  ? ??????  = 2 ? ??????





?	Quantization:
\textbullet 	Value discretization in steps of one LSB
\textbullet 	Infinite set of values ? finite set of values
\textbullet 	E.g., rounding
















?	Sample: S is closed, ??out = ??in
?	Hold: S is open, C holds ??out

?	IC1 with small Zout for fast charging of C
?	IC2 with high Zin for slow discharge of C
?	Switch: Transmission gate or single FET with small Ron



?	Aperture time:
time required for the 
switch to open
?	Droop: discharge 
of capacitor
?	Acquisition time: 
time to switch and 
charge the capacitor
?	Switching transients: 
voltage buffer ringing
?	How to design C?
\textbullet 	Large enough for small droop, small enough for fast charging
\textbullet 	Fast charging depends on ??on of the switch and ??out or ??out,max of IC1
\textbullet 	Sufficiently large for sufficiently small kT noise
C







?	Analog multiplexer use transmission gates which are bi-directional: 
MUX , DEMUX



?	Functionality
?	Static DAC errors
\textbullet 	Gain , offset error
\textbullet 	Differential non-linearity (DNL)
\textbullet 	Integral non-linearity (INL)
?	Dynamic DAC errors
\textbullet 	Jitter, Glitches
?	Types of DACs



?	Convert transient digital input signal to a transient analog output signal
?	Output voltage variation upon the variation of 1 LSB at the input ??LSB
?	Maximum (full scale) input voltage ??FS









VLSB


? Vref
2N



VFS



? Vref



2N ? 1
2N



?	Gain error:	Offset error:


?	Gain and offset errors do not affect linearity!
?	If needed, they are easy to compensate
\textbullet 	e.g., by pre-processing: scaling, shifting



?	Differential nonlinearity (DNL) is a measure of (non-) uniformity
?	The vector DNL(k) quantifies for each code k the deviation of the step 
size from the ideal (or average, in case of gain error) step size of 1 LSB



k
DNL(k)
000
N/D
001
-0.4
010
0.2
011
0.9
100
-0.7
101
0.9
110
-0.2
111
-0.7

Note: no gain and offset 
error in this example!



?	What if DNL(??) is smaller than ?1?

\textbullet 	The output value ??	is smaller than the previous sample ??(?? ? 1)

\textbullet 	Characteristic is non-monotonic!
?	Critical for feedback structures since non-monotonicity turns negative 
into positive feedback
?	Occurs most often when switching the MSB



?	Integral nonlinearity INL(k) is the deviation from the output value V(k) 
from the ideal output value Videal(k)


?	If


????NL


? 
????. 
???? 
then


DNL


? 
???? 
(mo
noto
nicit
y)







Note: no gain and offset 
error in this example!



?	Clock frequency/phase varies slightly
\textbullet 	Clock edges affected by jitter
\textbullet 	Uncertainty of the position of the clock edge
?	Rough estimation:
\textbullet 	Sinusoidal signal:

\textbullet 	Maximum change rate:

\textbullet 	Maximum jitter for error below one LSB:

?	10-bit DAC with 10 MSamples/s requires 30 ps accurate clock edges!



?	Glitches occur when 
turn-off and turn-on 
times are not precisely 
synchronized
?	S2 represents the MSB
output signal
?	S1 represents the
(?? ? 1) ? LSB output signal
?	Example: transition
S1 = 0111 to S2 = 1000



?	Resistor string DAC
?	Binary weighted resistive voltage divider DAC
?	Binary weighted current source DAC
?	Thermometer weighted DAC
?	R-2R ladder DAC
?	Pulse width modulation



?	Simple voltage divider


?	Inherently monotonic


?	Amount of resistors 
proportional to 2N
\textbullet 	3 bit ? 8 resistors
\textbullet 	8 bit ? 256 resistors
%?	Area ????W\_?????\_???2^??
\textbullet 	Resistors are large


Slide 19	Binary Weighted Resistive Voltage Divider DAC (1/2)
?	Inverting summing amplifier configuration








?	Each bit ???? produces an output voltage according to


?	Combined output:






?	Impractical for large N:
\textbullet 	Area ?? ?? W?? ? ???? ? 2??+1
\textbullet 	Resistors need to be very precise, RMSB must have the highest precision





?	Current mirror realization 
of the current sources:



?	Few number of resistors / current sources and switches
?	Large ratios of resistors, currents and switches (2N)
\textbullet 	Monotonicity not guaranteed
\textbullet 	5bit: 1uA, 2uA 4uA, 8uA, 16uA
\textbullet 	15uA ? 16uA means switch 4 sources off and 1 source on
\textbullet 	Assume statistical variations of 10% in each source
\textbullet 1u+2u+4u+8u < 16u is not guaranteed
\textbullet 	Linearity \textbullet Accuracy \textbullet Matching
?	Prone to glitches
\textbullet 	Large portions of the output are switched on/off



?	Current steering DAC:


?	Reduced glitches


?	Inherently monotonic


?	2N current sources



?	Binary to thermometer decoder needed:






?	Reduced area: ?? ?? W?? ? ????(3 ? ?? + 2)
?	The current ?? in every branch is given by:


?	Looking from node ?????? to the left or right one sees a resistance of 2??
The current flowing through each resistor splits half half








?	The current in the branch corresponding to bit ????
times split half half


is ???? + 1










?	R-2R ladder DAC: switches change from GND to -Vref
?	Solution: inverse R-2R network
\textbullet 	Switches change from GND to virtual GND
?	Working principle:



\textbullet 	??left

\textbullet 	??MSB


= 2?? ? ??left

= ? ??re??
2???


= ? ????????????
2???





??????????????
\textbullet 	??MSB?1 = ? 2???



LSB	MSB





??????????????
\textbullet 	??	= ? 2??? 
2???1



I
M
S
B
AN-1	AN-
2	A0
Z
l
e
f
t



?	From previous slide:



Vout,i



? ?di


? Vref


?2R ? ? R
2i



? Vref



? di


?

1

2
i

?
1














LSB	MSB







AN-1	AN-2


IMSB
A0
Zleft



?	Summing amplifier with weighted inputs:



































out


C	in1	C


in2


f	f



?	Transfer function from bit ???? to the output voltage:















cf. slide 19!




?	Combined output voltage:





























?	For N bit ? (2N - 1) + 2N unit capacitors ??unit = ??0/2??
\textbullet 	4 bit ? 15 + 16 = 31 unit capacitors ??0/16
?	Impractical for large N:
\textbullet 	Area ?? ?? 2 ? A??unit  ? 2??	with ????unit being the area of a unit capacitor
\textbullet 	Capacitors need to be very precise, CMSB must have the highest precision









MSB-Group



















?	Example: 8 bit ? 15 + 1 + 15 + 16 = 47 unit capacitors



?	Phase 1: reset




Right hand side for 10000000:


Right hand side in general:








?	MSB group all caps shorted between gnd and virtual gnd 
do not matter (i.e. do not store any charge)
?	From node ??2 to the output the circuit is a simple SC voltage amplifier


with a gain of


??out
??A2


????0
= ? 16
??0


= ? 1
16


?	Question? What is the voltage at node ??2?


Slide 34	Realization with less Capacitors \textbullet LSB Group (2/4)







V	? ? Vref	V	V 2    ? Vref	1





2	A	2


out ? ?	?	4


Slide 35	Realization with less Capacitors \textbullet  LSB Group (3/4)





V	? ?


V	? ? 1V



A2	C	4 ? 3C


4	ref	4	ref


0	0


V	V 2	Vref	1





out


? ?	?	?
16	4	24




Left hand side in general:





?	Pulse width modulated bitstream
?	Pulse width corresponds to the desired analog output voltage
?	RC low pass filter averages the PWM signal






?	Combination of various 
architectures
\textbullet 	optimize speed, linearity,
\textbullet 	glitches, matching


?	Commonly used for high 
performance DACs


?	Course DAC (MSBs) 
\textbullet  feeds\textbullet fine DAC (LSBs)



?	Each source supposed to have Iref
\textbullet 	Assume particular source has only 90% Iref
?	Calibration Phase
\textbullet 	VGS1 settles so that Q1 draws 10% Iref
?	Operation Phase
\textbullet 	Q1 still calibrates so that Id1 = Iref

	\end{multicols*}
\end{document}






























